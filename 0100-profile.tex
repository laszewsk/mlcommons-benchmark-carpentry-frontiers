\subsubsection{Profiling and Performance Analysis}
\label{sec:prof}

Profiling is the process of measuring a program's performance in
association with the locations in the source code in order to reveal
where resources (e.g., time and memory) are spent during execution.
Profiling is important in AI benchmarking for the following reasons:

\begin{itemize}
    \item Profiling helps explain why a particular method or
    implementation variant is faster than another.
    \item Profiling helps support fair and reproducible benchmarking.
    \item Profiling can distinguish between the essential computations
    and extraneous overheads.
    \item In a heterogeneous system, profiling can identify which
    components (e.g., CPU or GPU's CUDA cores vs. tensor cores) are
    being used by different parts of the application.
    \item Profiling can identify which specific library kernels are being used by different parts of the application.
\end{itemize}

Table~\ref{tab:merged_profiling_tools} provides a list of profiling tools that are useful for analysis of deep learning applications.


  \begin{table}

      \caption{Summary of Example Profiling Tools Useful for Deep Learning and AI Workloads}
\label{tab:merged_profiling_tools}

  ~\\
  This table and the refernces included in that table are located in the supplementary document.
  
  \end{table}

%\setlength{\tabcolsep}{2pt}
\renewcommand{\arraystretch}{1.1}

\onecolumn
\begin{landscape} 
{\tiny
  \begin{longtable}
{|p{
 0.25\textheight}|p{
 0.15\textheight}|p{
 0.2\textheight}|p{
 0.6\textheight}|}
  \caption{Summary of Example Profiling Tools Useful for Deep Learning and AI Workloads}
\label{tab:merged_profiling_tools}
\\ \hline
\rowcolor{blue!20}
\textbf{Tool / Category} & \textbf{Vendor / Maintainer} & \textbf{Level / Primary Use Case} & \textbf{Key Features and Capabilities} \\
\hline
\endfirsthead
\caption{Summary of Example Profiling Tools Useful for Deep Learning and AI Workloads (Cont.)} \\
\hline
\rowcolor{blue!20} \textbf{Tool / Category} & \textbf{Vendor / Maintainer} & \textbf{Level / Primary Use Case} & \textbf{Key Features and Capabilities} \\
\hline
\endhead
\hline
\multicolumn{4}{|r|}{{\footnotesize Continued on next page}} \\
\hline
\endfoot
\hline
\endlastfoot
\hline
% DATA
\rowcolor{gray!30} \multicolumn{4}{|l|}{\textbf{Framework Profilers}} \\ \hline
PyTorch Profiler~\citep{TensorFlow_System} & Meta & Framework-level & Records CPU/GPU activities, memory usage, and operator timings; integrates with TensorBoard and Perfetto; useful for training optimization and layer timing. \\ \hline
TensorBoard / TensorFlow Profiler~\citep{TensorFlow_System} & Google & Framework-level & Visualizes input pipelines, GPU kernels, and op-level timings; includes memory and device utilization tracing; supports bottleneck analysis. \\ \hline
torch.utils.bottleneck~\citep{PyTorch_System} & Meta & Framework-level & Combines autograd and Python profilers for quick bottleneck diagnostics. \\ \hline
JAX Profiler~\citep{JAX_Profiler} & Google & Framework-level & Works with TensorBoard to trace XLA compilation, HLO graphs, and TPU/GPU runtime performance. \\ \hline
NVIDIA DLProf~\citep{NVIDIA_DLProf} & NVIDIA & Framework-level (GPU-focused) & High-level view of deep learning layers and operations; integrates with TensorBoard DLProf plugin. \\ \hline
\rowcolor{gray!30} \multicolumn{4}{|l|}{\textbf{Hardware / System Profilers}} \\ \hline
Nsight Systems~\citep{NVIDIA_NsightSys_Doc} & NVIDIA & System-level & Timeline visualization of CPU–GPU interactions, kernel launch overheads, multi-process analysis, and NCCL tracing. \\ \hline
Nsight Compute~\citep{NVIDIA_NsightComp_Doc} & NVIDIA & Kernel-level & Detailed GPU kernel performance metrics: memory throughput, Tensor Core utilization, occupancy, and roofline analysis. \\ \hline
nvprof (deprecated)~\citep{NVIDIA:CUDA:ProfilerGuide} & NVIDIA & GPU-level & Legacy command-line CUDA profiler, replaced by Nsight tools. \\ \hline
VTune Profiler~\citep{Intel_VTune_Doc} & Intel & CPU/System-level & Hotspot analysis, vectorization, threading efficiency, and CPU performance bottlenecks. \\ \hline
omnitrace / rocprof / rocm-smi~\citep{AMD_ROCM_Doc} & AMD & GPU-level & Profiling and monitoring for AMD GPUs: kernel execution metrics, power, and temperature. \\ \hline
HPCToolkit~\citep{HPCToolkit_Paper} & Rice University & System-level (CPU+GPU) & Hierarchical performance profiling, time attribution to calling context, supports CUDA and HIP. \\ \hline
TAU~\citep{TAU_Paper} & University of Oregon & System-level (CPU+GPU+MPI) & Multi-level performance analysis, MPI integration, supports heterogeneous systems. \\ \hline
Perfetto~\citep{Perfetto_Google} & Google (Open Source) & System-level & High-resolution trace visualization, interoperable with PyTorch/TensorFlow profiler exports. \\ \hline
PAPI~\citep{PAPI_Paper} & University of Tennessee & Hardware counter interface & Provides access to CPU/GPU performance counters for integration with other profiling tools or custom instrumentation. \\ \hline
\rowcolor{gray!30} \multicolumn{4}{|l|}{\textbf{Compiler / Graph Profilers}} \\ \hline
XLA Profiler~\citep{XLA_Paper} & Google & Compiler-level (XLA) & Profiles XLA-compiled operations and execution times; supports JAX/TF and TPU/GPU workloads. \\ \hline
TorchDynamo / TorchInductor Debug Tools~\citep{TorchDynamo_TorchInductor} & Meta & Compiler-level (PyTorch 2.x) & Analyzes graph fusion, compiler optimizations, and operator performance of compiled PyTorch models. \\ \hline
Triton Profiler~\citep{Triton_Paper} & OpenAI & Kernel-level (Custom Kernels) & Reports kernel execution time, register usage, and occupancy for custom Triton GPU kernels. \\ \hline
\rowcolor{gray!30} \multicolumn{4}{|l|}{\textbf{Communication / Distributed Profilers}} \\ \hline
NCCL Profiler~\citep{NVIDIA_NCCL_Doc} & NVIDIA & Communication-level & Profiles NCCL collective communication operations (e.g., all-reduce, broadcast); timeline visualization of multi-GPU communication. \\ \hline
AWS SageMaker Debugger / Azure Profiler~\citep{AWS_SageMaker_Debugger,Azure_Profiler} & AWS / Microsoft & Cloud-level & Distributed GPU/CPU monitoring, training metric collection, and profiling at cloud scale. \\ \hline
Weights \& Biases, Comet, MLflow~\citep{Weights_Biases,Comet_ML,MLflow} & Multiple Vendors & Experiment / Cloud-level & Logs performance traces, GPU utilization, integrates with PyTorch and TensorFlow profilers for real-time monitoring. \\ \hline
\rowcolor{gray!30} \multicolumn{4}{|l|}{\textbf{System \& Memory Profilers}} \\ \hline
Torch / TensorFlow Memory Tools~\citep{Torch_Tensorflow_Memory} & Meta / Google & Framework-level (Memory) & Reports GPU memory allocation, fragmentation, and utilization trends for debugging memory bottlenecks. \\ \hline
  Python Profilers (cProfile, py-spy)~\citep{Python_Profilers} & Python Community & CPU-level & Measures Python-level overhead and I/O performance; used for diagnosing data preprocessing bottlenecks. \\ \hline

\end{longtable}
}
    
\end{landscape}


\clearpage

It is important to note that the tooling and services exist for
supporting different levels of infrastructures. This includes examples
for framework-level, system-level (including CPU and GPU), kernel-level,
compiler-level, communication-level, and cloud-level.

Furthermore, we aim here to provide comprehensive coverage of the AI
profiling stack, which affords users the insights into cross-vendor and
cross-platform capabilities and offerings, and also provide key analysis
of features of the said tools and services.

We believe it is essential to increase awareness and use of profiling
tools through AI benchmarking efforts, enabling a better understanding
of bottlenecks in AI applications. Additionally, we need to educate the
community about policy limitations that may implicitly restrict specific
profiling tools. As discussed previously, one such policy restriction is
that not all profiling information is available for energy benchmarks.
Such restrictions may also be in place for additional hardware profiling
measures.

Lastly, we need to educate the community about the {\em performance
impact} of profiling costs to avoid over-profiling. Therefore, it makes
sense that AI benchmarks should be able to choose the level of profiling
selectively. This information is vital to support the FAIR principles
and ensure that benchmarks are comparable.
