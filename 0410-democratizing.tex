%
% FROZEN
% FROZEN
% FROZEN
% FROZEN
% FROZEN
%

\section{Towards AI Benchmark Democratizing}
\label{sec:dem}

Our goal is to make AI benchmarking transparent, reproducible, and community-driven. Democratization empowers a broader range of participants to contribute to and learn from AI performance evaluation.

Introducing democratization tools, datasets, and evaluation frameworks that are openly accessible and easy to use can allow anyone—from students to independent researchers—to measure, compare, and improve AI models. 

One of the biggest hurdles we find is that some benchmarks, probably rightfully so, target hyperscale or leadership-class machines. However, in order to increase the community and raise awareness, smaller scale benchmarks need to be available.

As such, the following aspects can improve democratization:

\begin{itemize}
    \item \textbf{Accessibility:}
    \begin{itemize}
        \item Benchmarks, datasets, and tooling ought to be open-source or freely available.
        \item Users may not need to rely on expensive hardware or proprietary software to participate.
        \item Examples can be leveraged to develop new benchmarks. One can start with examples provided by MLCommons open datasets, pre-built benchmarking pipelines, and Jupyter notebooks with ready-to-run benchmarks.
    \end{itemize}

    \item \textbf{Usability:}
    \begin{itemize}
        \item Interfaces, documentation, and examples in existing efforts can serve as starting point to developing user-friendly, allowing non-experts to run benchmarks.
        \item Providing automated scripts and tutorials reduces the barrier to entry.
    \end{itemize}

    \item \textbf{Transparency:}
    \begin{itemize}
        \item Specifying clear definitions of metrics, scoring methods, and evaluation procedures ensures everyone understands the results.
        \item Improved transparency addresses the hide everything in a “black box” approach,  where only insiders can interpret outcomes.
    \end{itemize}

    \item \textbf{Community Participation:}
    \begin{itemize}
        \item Anyone with minimal but sufficient knowledge should be able to contribute to benchmarks, improve tools, or submit models.
        \item Democratization also means encouraging collaboration and reproducibility across institutions and geographies (e.g., engaging the broader community). 
    \end{itemize}

    \item \textbf{Impact:}
    \begin{itemize}
        \item Through democratization, smaller teams or educational institutions can contribute and benefit from learning, competing, and comparing AI benchmarks.
        \item Through democratization, fairness and innovation is fostered because knowledge and evaluation methods are disseminated.
    \end{itemize}
\end{itemize}


