\section{Conclusion}
\label{sec:conclusion}

Overall, this comprehensive paper has explored the motivations and pathways for creating a holistic benchmark  carpentry effort, paying specific attention to aspects that can democratize AI benchmarks.
This was achieved by (a) providing standardized and formal definitions of benchmarks, and (b) identifying a representative set of benchmarks related to AI activities.
Finally, we propose an AI Benchmark Carpentry curriculum that integrates the various topics discussed into a structured learning activities 
to empower  
practitioners with reproducible coding practices, experiment‑management 
skills, and an ethical lens on benchmarking. By embedding FAIR principles, 
bias‑mitigation techniques, and performance‑tuning modules, the curriculum 
offers a scalable pathway for communities—from academic labs to industry 
R\&D—to build, share, and improve benchmarks in a collaborative, 
transparent manner.

Together, these activities foster democratization of AI benchmarks and can be utilized to grow the community and the understanding on how benchmarks may effect an individual activity or even community. While deploying such activities, we hope to grow community awareness and overcome the lack of well defined activities to educate the community in this regard. While fostering these activities we also address the need for more easily develop dynamic and adaptable benchmarks.