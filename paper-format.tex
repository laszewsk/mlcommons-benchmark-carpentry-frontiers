% -----------------------------------------------------
% Encoding & Fonts
% -----------------------------------------------------
\usepackage[utf8]{inputenc}
\usepackage[T1]{fontenc}

\usepackage{lmodern}
\usepackage{textcomp}
\usepackage{microtype}
\usepackage{amsmath}

\usepackage{caption}
\captionsetup{textfont=normalfont}

% -----------------------------------------------------
% Math & Symbols
% -----------------------------------------------------
\usepackage{amsmath, amssymb, amsfonts}
\usepackage{bm} % Bold math
\usepackage{siunitx}
\sisetup{per-mode=symbol}
\DeclareSIUnit{\COtwoe}{\ensuremath{\mathrm{CO_2e}}}
\DeclareSIUnit{\gCOtwoe}{\ensuremath{\text{g\,CO_{2e}}}}
\DeclareSIUnit{\cent}{\text{\textcent}}

% -----------------------------------------------------
% Tables
% -----------------------------------------------------
\usepackage{pdflscape}   %% 
\usepackage{booktabs}    % Professional tables
\usepackage{tabularx}    % Flexible width tables
\usepackage{array}       % Extended table formatting
\usepackage[table]{xcolor}
\usepackage{longtable}
\usepackage{makecell}

% -----------------------------------------------------
% Listings & Code
% -----------------------------------------------------
\usepackage{listings}
\usepackage{xcolor}
\lstdefinestyle{pystyle}{
  language      = Python,
  basicstyle    = \ttfamily\small,
  keywordstyle  = \color{blue!70!black}\bfseries,
  commentstyle  = \color{green!50!black},
  stringstyle   = \color{orange!70!black},
  numbers       = left,
  numberstyle   = \tiny,
  numbersep     = 6pt,
  frame         = single,
  rulecolor     = \color{black!40},
  tabsize       = 4,
  showstringspaces = false,
  captionpos    = b
}

% -----------------------------------------------------
% Lists & Custom Enums
% -----------------------------------------------------
\usepackage{enumitem}
\newlist{goal}{enumerate}{1}
\setlist[goal,1]{label=\textbf{G\arabic*},leftmargin=1.8em,align=left}

% -----------------------------------------------------
% Graphics, TikZ & Diagrams
% -----------------------------------------------------
\usepackage{graphicx}
\graphicspath{{ontology/images/}{images}}
\usepackage{tikz}
\usetikzlibrary{positioning,arrows.meta,shadows,trees,mindmap}
\usepackage{forest}
\usepackage{smartdiagram}
\usepackage[framemethod=tikz]{mdframed}

% Forest tree style
\forestset{
  skan tree/.style={
    for tree={
      drop shadow,
      text width=3cm,
      grow'=0,
      rounded corners,
      draw,
      top color=white,
      bottom color=blue!20,
      edge={Latex-},
      child anchor=parent,
      anchor=parent,
      tier/.wrap pgfmath arg={tier ##1}{level()},
      s sep+=2.5pt,
      l sep+=2.5pt,
      edge path'={
        (.child anchor) -- ++(-10pt,0) -- (!u.parent anchor)
      },
      node options={ align=center },
    },
    before typesetting nodes={
      for tree={
        content/.wrap value={\strut ##1},
      },
    },
  },
}

% -----------------------------------------------------
% Hyperlinks & URLs
% -----------------------------------------------------
\usepackage[hyphens]{url}
\usepackage{hyperref}
\renewcommand{\UrlFont}{\footnotesize}

% -----------------------------------------------------
% Miscellaneous
% -----------------------------------------------------
\usepackage{comment}
\usepackage{stfloats}     % For positioning floats at bottom
\usepackage{adjustbox}    % For scalable content
\usepackage{todonotes}    % For todo notes

% -----------------------------------------------------
% Custom Commands
% -----------------------------------------------------
\newcommand{\sysfs}{\texttt{/sys/class/powercap}\xspace}
\newcommand{\code}[1]{\texttt{#1}}
\newcommand{\textscit}[1]{\text{\scshape\itshape#1}}
\newcommand{\ngreen}{bottom color=green!20}
\newcommand{\ngrey}{bottom color=gray!20}
\newcommand{\nred}{bottom color=red!20}
\newcommand{\nwhite}{bottom color=white!20}
\newcommand{\TODO}[1]{\todo[inline]{{\footnotesize #1}}}
\newcommand{\COMMENT}[2]{\todo[inline,color=green!20!white]{{\footnotesize Comment from #1: #2}}}
\newcommand{\GVL}[1]{{\color{blue} #1}}
\newcommand{\REPLACE}[2]{{\color{red}\it #1} \begin{quote}{\color{blue}#2}\end{quote}}

% checkmarks

\definecolor{darkgreen}{rgb}{0.0, 0.5, 0.0}

\newcommand{\YES}{\textcolor{darkgreen}{\checkmark}}
\newcommand{\NO}{\textcolor{red}{$\times$}}
\newcommand{\LINK}{$\hookrightarrow$}

%\newcommand{\FILE}[1]{\parindent 0pt\textcolor{blue}{File: #1}}
\definecolor{lightblue}{rgb}{0.7,0.85,1.0}

\newcommand{\FILE}[1]{%
  \parindent0pt%
  \todo[inline,backgroundcolor=lightblue!60]{{\footnotesize File: #1}}%
}

% Define \FINPUT macro: prints filename via \FILE, then inputs the file
\newcommand{\FINPUT}[1]{%
  % \FILE{#1}%
  
  \input{#1}%
}
% Optional if needed:
% \usepackage{float}       % If you need [H] positioning

% THIS IS A BUG REMOVE THE UNICODE 
\DeclareUnicodeCharacter{2082}{\textsubscript{2}}


% DYNAMIC SECTIONS

\makeatletter

% Detect current sectioning depth:
% 0 = none yet
% 1 = section
% 2 = subsection
% 3 = subsubsection
% 4 = paragraph
% 5 = subparagraph
\newcommand{\getcurrentlevel}{%
  \ifnum\c@subparagraph>0 5%
  \else\ifnum\c@paragraph>0 4%
  \else\ifnum\c@subsubsection>0 3%
  \else\ifnum\c@subsection>0 2%
  \else\ifnum\c@section>0 1%
  \else 0%
  \fi\fi\fi\fi\fi
}

% Apply section command corresponding to a numeric level (capped at subparagraph)
\newcommand{\uselevel}[2]{%
  \ifcase#1%
    \section{#2}% 0
  \or \section{#2}% 1
  \or \subsection{#2}% 2
  \or \subsubsection{#2}% 3
  \or \paragraph{#2}% 4
  \or \subparagraph{#2}% 5
  \else \subparagraph{#2}% fallback beyond max depth
  \fi
}

% Generic relative-level helper
\newcommand{\usedynlevel}[2]{%
  \edef\level{\number\numexpr\getcurrentlevel+#1\relax}%
  \ifnum\level>5 \def\level{5}\fi % cap at subparagraph
  \uselevel{\level}{#2}%
}

% Public dynamic commands
\newcommand{\dynsection}[1]{\usedynlevel{1}{#1}}        % one deeper
\newcommand{\dynsubsection}[1]{\usedynlevel{2}{#1}}     % two deeper
\newcommand{\dynsubsubsection}[1]{\usedynlevel{3}{#1}}  % three deeper
\newcommand{\dynparagraph}[1]{\usedynlevel{4}{#1}}      % four deeper

\makeatother

\setcounter{tocdepth}{10} 
