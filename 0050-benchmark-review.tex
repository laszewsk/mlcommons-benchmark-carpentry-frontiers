\section{Review of Benchmark Related to this Effort}
\label{sec:benchmarks}

This section provides an overview of key benchmarking efforts that motivated our paper. We start with HPC benchmarks and also address MLCommons benchmark efforts.

\subsection{HPC Benchmarking}
\label{sec:benchmarks-hpc}
\label{sec:hpc}

HPC benchmarking has a great impact on the activities that we report here and we can learn a lot from these efforts. Some of the most known efforts are TOP500 and Green500.


\subsubsection{TOP500}
\label{sec:benchmarks-hpc-top500}

The list of world's largest supercomputers has been released biannually
for nearly 4 decades now and thus offers a number of important lessons
in designing sustainable benchmarks. At the heart of the TOP500 scoring procedure, which yields a ranked list of 500 supercomputing installations, is the LINPACK benchmark~\cite{dongarra2003hpl}, which bears the name of the
namesake software library~\cite{dongarra1979linpack} for
solving systems of linear equations. This linear solver package was
designed in the 1970s and implemented in FORTRAN. The user guide for the
library was published in 1979 and included a list of only 24
computers~\cite{dongarra1979linpack}. The following decades brought in various aspects of scaling into the
software, the list sizes, and the machines submitted for inclusion in the ranking as well as data and reporting information.

\subsubsection{Green500}
\label{sec:benchmarks-hpc-green500}

Power and energy play a dominant role in the modern world of high-performance and distributed computing, with multi-megawatt data centers and computing facilities abound in many locations across the globe.
The issues of excessive power draw and energy consumption
data in the mid-2000s~\cite{feng2005pwrprofsciapps,
cameron2005hpcpowerdistcompsciapps} culminated in a special working
group of cross-industry members~\cite{specpower2008, specpower}, combining the TOP500 ranking with the available power
draw information from the supercomputers to yield the ranking
called Green500~\cite{green500}. Since then, it is published
alongside the TOP500 ranking and continues to underscore the
importance of efficient energy use at large HPC installations.

\subsubsection{HPC innovation}
\label{sec:benchmarks-hpc-innov}

Besides the recognition of development of tools and software to
facilitate the use of HPC systems and foster democratization, power
consumption monitoring has been integrated at the various levels of HPC
facilities, from the processing and networking elements to the data
center level infrastructure. Also, by utilizing different floating-point
precisions~\cite{abdelfattah2021mxpsurvey} the applications improve
their efficiency and benefit from a great impact on the system
performance due to direct targeting of the specific architectural
designs.

The creation of leaderboards has led to a better understanding of the overall HPC system, but insights can be limited by misalignment of algorithm scaling and leaderboard projections.
To counter misalignment, benchmarks should closely resemble the scientific task to be benchmarked.
In some cases, it is informative to include end-to-end performance, including data storage limitations.



\subsection{Machine Learning Benchmarks}
\label{sec:ai-benchmakrs}
\label{sec:benchmarks-mlcommons}

Benchmarking in scientific machine learning (ML) has emerged as a critical area to guide algorithm development, enable fair comparisons towards progress and innovation, and facilitate reproducibility. The development of ML benchmarks for science is especially critical because of the multi-disciplinary nature of the development, often including domain experts, computing hardware developers, and ML researchers.  That, coupled with the variety of tasks and workloads, makes {\it high quality} benchmarking critical to making progress. 

To obtain an overview how many academic benchmarks have been published in well known public domain archives, so we queried arXiv~\cite{www-arXiv} and Google Scholar~\cite{www-google-scholar}. Note that according to Google, Google Scholar does not include all entries from arXiv, but it does include most of them. However, it also includes many more resources, so we expect a larger number from Google Scholer. As of Oct 1, 2025, we find 106 entries on arXiv when searching for the topic {\em ``AI benchmark''}.
executing equivalent queries in Google Scholar yields 2,490 entries for {\em ``AI benchmark''}. It is evident from this that a complete survey of these papers is difficult to achieve through manual inspection. In an upcoming effort, we plan to explore how to automatically categorize these entries using LLMs while implementing an agentic AI framework for it.

The vast number and diversity of scientific tasks poses challenges to finding a well-defined, high-quality benchmark for any given task.
To improve discoverability, we have cataloged in this paper all MLCommons benchmarks that have a result submission.
Secondly, we have developed an ontology \cite{www-las-mlcommons-benchmark-collection,www-mlcommons-science-benchmarks-paper} that allows users to identify suitable benchmarks.

\subsubsection{MLCommons}

MLCommons \cite{www-mlcommons} provides one of the most comprehensive and standardized ecosystems of AI benchmarking. It addresses training, inference, scientific computing, and domain-specific benchmarks. Most prominently, the MLPerf benchmark suite—covering datacenter, edge, mobile, and training applications—establishes industry-wide baselines for performance, accuracy, power efficiency, and quality of service across diverse model classes such as computer vision, language, recommendation, speech, and reinforcement learning. Additionally, it offers specialized evaluations including MLPerf Tiny for microcontroller-class devices, MLPerf Storage for I/O workloads, and MLPerf Science for large-scale scientific AI. Furthermore, MLCommons promotes the reproducibility through initiatives such as Croissant ML, a standardized metadata schema for datasets, and MLCube, a portable container-based model packaging standard. Additional domain-specific working groups in medical AI, multilingual speech, and responsible AI have recently expanded the targeted domains. 

We have provided a comprehensive list of benchmarks in Tables \ref{tab:benchmarks-mlcommons} and \ref{tab:llm_benchmarks_long}.  The tables contain information about the benchmark name, model, task, domain, model type, metrics, hardware, and a brief note. The evaluations of the AILuminate benchmarks can be found on the MLCommons Web pages and include (a) Safety / Jailbreak Tests, (b) LLM Safety Evaluation, (c)  Responsible AI / Alignment (d)  LLM (Decoder) (e)  Safety Rate, Toxicity Score (f) Cloud LLM APIs (g) Robustness and Alignment.

\subsubsection{Ontology}

To improve discoverability of suitable benchmarks for a given task, we introduce a definition and AI Benchmark ontology of scientific machine learning benchmarks, where benchmarks are classified and mapped to their scientific domain and machine learning task type in~\cite{www-mlcommons-science-benchmarks-paper}. This work grew out of the Web page created at  \cite{www-las-mlcommons-benchmark-collection}, \cite{www-mlcommons-benchmarks} and provides an easy to use interactive mechanism to query the cataloged benchmarks.

New AI benchmarks are added through an open submission workflow overseen by the MLCommons Science Working Group. Each submission is evaluated against a rubric of currently six categories (Software Environment, Problem Specification, Dataset, Performance Metrics, Reference Solution, Documentation) that assigns an overall rating and potential endorsement. The scoring framework enables stakeholders, researchers, domain scientists, and hardware vendors to identify representative subsets of benchmarks that align with their specific priorities. The ontology supports adding new scientific domains, AI/ML motifs, and computing motifs.   

A subset of information collected by the Web page is shown in Table \ref{tab:ontology}. It not only includes some elementary information about the benchmarks but also a perceived rating displayed as a radar chart. Such radar charts include ratings from 1-5, where 5 is the best rating. Ratings are identified for documentation, specification, software, metrics, dataset, and reference solution. The Web page not only includes an automatically generated report of all benchmarks in PDF format, but also a convenient online publication of the benchmarks with convenient search capabilities.

\begin{comment}
To initiate this effort, we issued queries to ChatGPT and obtained an initial list of benchmarks, as shown in Table \ref{tab:scientific_ai_benchmarks}. In addition, we identified several benchmarks based on the interests of members of the MLCommons Science and HPC Working Groups. This includes a categorization of MLCommons benchmarks according to working group interests.
\end{comment}

\setlength{\tabcolsep}{2pt}
\renewcommand{\arraystretch}{1.1}

{\tiny
\onecolumn
\begin{landscape} 
\begin{longtable}
{|p{
0.1\textwidth}|p{
0.09\textwidth}|p{
0.1\textwidth}|p{
0.2\textwidth}|p{
0.2\textwidth}|p{
0.2\textwidth}|p{
0.2\textwidth}|p{
0.2\textwidth}|}
\caption{MLCommons Benchmarks}
\label{tab:benchmarks-mlcommons}
\\ \hline
\rowcolor{blue!30}
\textbf{Benchmark Name} & \textbf{Model} & \textbf{Task} & \textbf{Application Domain / Use Case} & \textbf{Model Type / Architecture} & \textbf{Metrics / KPIs} & \textbf{Hardware} & \textbf{Notes / Description} \\
\hline
\endfirsthead
\caption{MLCommons Benchmarks (Cont.)} \\
\hline
\rowcolor{blue!30}
\textbf{Benchmark Name} & 
\textbf{Model} & 
\textbf{Task} & 
\textbf{Application Domain / Use Case} & 
\textbf{Model Type / Architecture} & 
\textbf{Metrics / KPIs} & 
\textbf{Hardware} & 
\textbf{Notes / Description} \\
\hline
\endhead
\hline
\multicolumn{8}{|r|}{{\footnotesize Continued on next page}} \\
\endfoot
\hline
\endlastfoot
\hline
%
% INFERENCE: DATACENTER
%
\rowcolor{gray!20} \multicolumn{8}{|l|}{\textbf{MLPerf Inference: Datacenter}} \\ \hline
 deepseek-r1 & DeepSeek R1 (671B params) & Reasoning / Code Generation & Knowledge \& Reasoning, Complex Problem Solving, Step-by-Step Planning & Large Language Model (LLM), Reasoning LLM, High context/output length (up to 20K tokens) & Accuracy: Exact Match, Code Evaluation; Latency: TTFT (Time to First Token), TPOT (Time Per Output Token)& Data Center GPUs (NVIDIA H100/H200) with massive VRAM, optimized for $671$B parameters. &The model's large output length emphasizes its use in complex reasoning chains. Requires powerful systems (e.g., multiple H100 GPUs). \\ \hline
 dlrm-v2-99 & DLRM-v2 & Recommendation & Personalized product/content recommendation (e.g., e-commerce, social media feeds) & Deep Learning Recommendation Model (DLRM), Sparse/Dense Architecture & Throughput: Queries Per Second (QPS); Latency: 99th Percentile Latency & Data Center CPUs and GPUs (NVIDIA B200/GB200/B300), prioritizing high I/O and memory bandwidth for massive embedding tables. &Tests high-throughput, low-latency deployment for online services with a 99\% latency constraint. \\ \hline
 dlrm-v2-99.9 & DLRM-v2 & Recommendation & Personalized product/content recommendation (e-commerce, social media feeds) & Deep Learning Recommendation Model (DLRM), Sparse/Dense Architecture & Throughput: Queries Per Second (QPS); Latency: 99.9th Percentile Latency &Data Center CPUs and GPUs (NVIDIA H200), often using higher precision to ensure quality target is met. &Tests high-throughput, very low-latency deployment for critical online services with a strict 99.9\% latency constraint. \\ \hline
 llama2-70b-99 & Llama 2 (70B params) & Large Language Model (LLM) Inference & General text generation, chat, summarization, and understanding & LLM, Transformer-based & Throughput: Tokens Per Second (TPS); Latency: TTFT, TPOT (99th Percentile) & Data Center GPUs (e.g., AMD MI300X/MI325X, NVIDIA B200/GB200/H100/H200/L40S, MS-Intel Arc Pro B60) in multi-GPU configurations, focused on high throughput and low latency. &Represents a larger LLM workload, measuring performance under a 99\% latency constraint. \\ \hline
 llama2-70b-99.9 & Llama 2 (70B params) & Large Language Model (LLM) Inference & General text generation, chat, summarization, and understanding & LLM, Transformer-based & Throughput: Tokens Per Second (TPS); Latency: TTFT, TPOT (99.9th Percentile) & Data Center GPUs (AMD MI300X/MI325X, NVIDIA B200/GB200/H100/H200/L40S, MS-Intel Arc Pro B60), often testing the limits of precision vs. speed trade-offs. &Represents a larger LLM workload, measuring performance under a stricter 99.9\% latency constraint. \\ \hline
 llama3.1-8b-datacenter & Llama 3.1 (8B params) & Summarization / Text Generation & Low-cost, high-volume LLM services, interactive code assistants & LLM, Transformer-based & Accuracy: ROUGE metrics (1, 2, L); Latency: TTFT $\le$2s, TPOT $\le$100ms (Server) & Single-node systems or smaller GPU clusters, used to lower the entry barrier for the MLPerf Training suite. &Benchmarks a smaller LLM for efficient deployment in both Data Center and Edge scenarios. \\ \hline
 llama3.1-405b & Llama 3.1 (405B params) & Large Language Model (LLM) Inference & Generative AI, high-capability models & LLM, Transformer-based & Throughput: Output Tokens per second; Latency: TTFT, TPOT &Large-scale AI Clusters and Supercomputers (requires hundreds of GPUs (NVIDIA B200/GB200/GB300/H100/H200) with high-speed interconnects). &One of the largest LLMs in the suite, demonstrating the need for advanced parallelism (tensor, pipeline) on high-end systems (e.g., NVIDIA H200). \\ \hline
 mixtral-8x7b & Mixtral (46.7B total params) & Large Language Model (LLM) Inference & generative AI, multilingual tasks & Mixture-of-Experts (MoE) LLM (activates $\approx$13B params per token) & Throughput: Tokens Per Second (TPS); Latency &Data Center GPUs (AMD MI300X/MI325X, NVIDIA H200/RTX PRO 6000), optimizing MoE architecture for low active compute per token. &Showcases the efficiency of MoE architecture, offering high quality with lower active compute cost than dense models. \\ \hline
 retinanet & Retinanet-ResNext50 & Object Detection & Identifying and localizing objects in images & Object Detection Model, often with ResNext backbone and FPN & Accuracy: mAP (mean Average Precision); Throughput: Samples Per Second &Data Center and Edge GPUs (NVIDIA GeForce RTX 4090/H200/L4-PCIe/L40S), measuring both throughput and latency under a $100$ms constraint. &A standard computer vision benchmark using the OpenImages dataset. \\ \hline
 rgat & Relational Graph Attention Network & Node Classification & Graph data analysis, social network processing, knowledge graphs & Graph Neural Network (GNN), Graph Attention Network (GAT) variant & Accuracy (on node classification); Throughput: Samples Per Second & Data Center GPUs (NVIDIA B200), specifically testing performance on irregular, graph-structured data.&Addresses graph-structured data and multi-relational graphs, testing system efficiency for complex graph workloads. \\ \hline
 stable-diffusion-xl & Stable Diffusion XL (SDXL) & Text-to-Image Generation & Generative AI for creating high-quality images from text prompts & Diffusion Model (Latent Diffusion) & Throughput: Images Per Second; Latency & Data Center and Professional GPUs (AMD MI325X,NVIDIA B200/H100/H200/L4-PCI/L40S/NVIDIA RTX PRO 6000), focusing on the speed of image generation (samples/second).&Represents the Text-to-Image Generative AI domain, measuring the speed of image synthesis. \\ \hline
 whisper & Whisper-Large-V3 & Automatic Speech Recognition (ASR) & Converting spoken audio to text & Encoder-Decoder Transformer, Speech-to-Text Model & Accuracy: WER (Word Error Rate), Word Accuracy (Acc); Latency & Data Center GPUs (NVIDIA B200/GB200/GeForce RTX 4090/H100/H200/L4-PCIe/L40S), measuring performance on a complex sequence-to-sequence model for speech.&An ASR benchmark on multilingual audio, measuring both encoder (audio feature) and decoder (token generation) performance. \\ \hline
%
% HPC
%
\rowcolor{gray!20} \multicolumn{8}{|l|}{\textbf{MLPerf HPC}} \\ \hline
 CosmoFlow & CosmoFlow 3D CNN & Regression & Astrophysics, Cosmology (predicting properties of the universe from simulation data) & 3D Convolutional Neural Network (3D CNN) & Time to Quality (TTQ) (e.g., Time to reach validation MAE $\le 0.124$) & Supercomputers \& Large HPC Clusters (e.g., Fugaku, Perlmutter). Stresses distributed training, 3D data handling, and fast data I/O for massive volumetric datasets ($\approx 5$ TB) GPUs used for running this benchmark: NVIDIA A100/V100.&Uses massive 3D volumetric data ($\approx 5.1$ TB). Stresses memory bandwidth and interconnect.  \\
\hline
 DeepCAM & DeepCAM Encoder-Decoder & Semantic Segmentation & Climate Science, Extreme Weather Prediction (identifying atmospheric rivers, tropical cyclones) & Convolutional Encoder-Decoder (e.g., U-Net or DeepLab-like) & Time to Quality (TTQ) (e.g., Time to reach validation IoU $\ge 0.82$) & Supercomputers \& Large HPC Clusters. Stresses large-scale image processing, high-dimensional data (many channels), and efficient communication on systems with thousands of GPUs (A100/P100/V100). &Trained on massive, high-resolution 2D image data ($\approx 8.8$ TB). Stresses I/O and communication efficiency. \\
\hline
 OpenCatalyst & DimeNet++ & Regression & Computational Chemistry, Materials Science (discovering new catalysts for energy storage) & Graph Neural Network (GNN) & Time to Quality (TTQ) (Time to reach target energy/force prediction error) & Supercomputers \& Large HPC Clusters. Stresses performance on graph-structured data (atomic systems) and complex GNN operations that require high GPU utilization. GPUs used for running this benchmark: NVIDIA A100/P100/V100. & Models atoms and bonds as a graph structure. Benchmarks complex, irregular GNN workloads at scale.  \\
\hline
%
% TRAINING
%
\rowcolor{gray!20}\multicolumn{8}{|l|}{\textbf{MLPerf Training}} \\ \hline
 BERT (Bidirectional Encoder Representations from Transformers) & NLP - Question Answering & General NLP, Text Understanding & Transformer (Encoder) & Time to Quality (TTQ) (F1 Score on SQuAD) & Data Center GPUs, Accelerators & CPU, Single GPU (e.g., NVIDIA A100/H100), or moderate clusters. &A foundational benchmark for Natural Language Processing tasks. \\
\hline
 DLRM-dcnv2 (Deep Learning Recommendation Model - DCNv2)& Recommendation Systems & E-commerce, Content Streaming, Personalized Ads & Deep Learning Recommendation Model w/ DCNv2 & Time to Quality (TTQ) ($\text{AUC}$ on Criteo 4TB) & Data Center GPUs, Specialized Accelerators &Large-scale GPU clusters with high-speed interconnects (e.g., InfiniBand) for distributed training. This benchmark was running on GPUs: NVIDIA B300/B200/GB200/H200/H100/H200.&Stresses memory bandwidth and communication for massive embedding tables. \\
\hline
 llama2-70b-lora & LLM Fine-Tuning & Customizing LLMs for specific enterprise tasks & Transformer with LoRA & Time to Quality (TTQ) (ROUGE Score) & Multi-GPU servers, Mid-size GPU clusters & High-end Multi-GPU servers or small clusters (e.g., systems with AMD MI300X/MI325X/MI350X/MI355X, NVIDIA B200/B300/H100/H200). &Measures the efficiency of Low-Rank Adaptation (LoRA) on a $\approx 70\text{B}$ parameter model. \\
\hline
 llama3.1-405b & LLM Pretraining & Generative AI, Foundational Model Development & Transformer-based LLM ($\approx 405\text{B}$ params) & Time to Quality (TTQ) (Log Perplexity) & Large-scale, Multi-node GPU clusters & Single Node or small GPU systems (e.g., a few GPUs per node) to keep the benchmark accessible.Benchmark running on GPUs: NVIDIA B200/B300/H200.&The largest, most compute-intensive benchmark for pretraining state-of-the-art LLMs.  \\
\hline
 RetinaNet & Object Detection & Autonomous Vehicles, Surveillance, Image Analysis & One-stage Object Detector (ResNet, FPN) & Time to Quality (TTQ) ($\text{mAP}$ on COCO) & Data Center GPUs, Cloud Instances &Single or multi-GPU systems (NVIDIA B200/H200/RTX Pro 6000), often used in both Datacenter and Edge devices for inference.&Measures performance for a core computer vision task: localizing and classifying objects. \\
\hline
 RGAT (Relational Graph Attention Network) & GNN - Node Classification & Drug Discovery, Social Network Analysis, Fraud Detection & Relational Graph Attention Network (R-GAT) & Time to Quality (TTQ) (Accuracy on IGBH) & Systems optimized for high-bandwidth interconnects & GPU-based systems (NVIDIA B200/B300/H100), optimized for workloads with complex, sparse data structures like graphs.&Focuses on the irregular memory access and communication patterns of Graph Neural Networks. \\
\hline
 Flux1 (stable-diffusion) & Text-to-Image Generation & Generative AI, Digital Art, Content Creation & Latent Diffusion Model (U-Net, Transformer) & Time to Quality (TTQ) ($\text{FID}$ and $\text{CLIP}$ Scores) & Multi-GPU servers, Cloud Instances & High-performance Single or Multi-GPU systems (especially for fast inference or training). This benchmark was running on: NVIDIA B200/GB200/GB300.&Benchmarks the training of a major generative model in the AI industry. \\
\hline
%
% Inference:Edge
%
\rowcolor{gray!20}\multicolumn{8}{|l|}{\textbf{MLPerf Inference: Edge}} \\ \hline
 3D U-Net (99\%) & 3D U-Net & Medical Image Segmentation & Healthcare, Volumetric Imaging (e.g., MRI/CT) & 3D Convolutional Encoder-Decoder CNN & Accuracy (Dice Score), Latency, Throughput (QPS) & Data Center GPUs (e.g., NVIDIA A100/H100), high-performance computing (HPC) systems, specialized accelerators. &99\% of reference accuracy target. Typically runs in Offline scenario for batch processing of medical scans. \\
\hline
 3D U-Net (99.9\%) & 3D U-Net & Medical Image Segmentation &Healthcare, High-Fidelity Imaging & 3D Convolutional Encoder-Decoder CNN & Accuracy (Dice Score), Latency, Throughput (QPS) & Data Center GPUs (e.g., NVIDIA A100/H100), high-performance computing (HPC) systems, specialized accelerators. &99\% of reference accuracy target. Represents a stricter quality constraint, often requiring higher-precision compute (e.g., FP16 vs. INT8). \\
\hline
 llama3.1-8b-edge & Llama 3.1 (8B params) & Text Generation / Summarization & Edge AI, On-device LLMs, Interactive Assistants & Quantized Transformer (Decoder-only LLM) & Tokens Per Second (TPS), Latency (TTFT, TPOT), Power & Edge devices, mobile SoCs (System-on-Chips), smaller GPUs (MS-Intel Arc Pro B60), high-end CPUs. &Benchmarks a modern, smaller LLM variant optimized for performance and low-latency on resource-constrained Edge devices. \\
\hline
 resnet & ResNet50-v1.5 & Image Classification & Vision, Quality Control, Surveillance	 & CNN (Residual Network) & Accuracy (Top-1), Latency, Throughput (QPS) & Data Center GPUs (NVIDIA GeForce RTX 4090/RTX-2000E), Edge devices, Mobile SoCs, CPUs, specialized accelerators. &The foundational computer vision benchmark, often used as a baseline for measuring performance and efficiency across all MLPerf tiers. \\
\hline
 retinanet & RetinaNet-ResNext50 & Object Detection & Autonomous Vehicles, Advanced Security Systems	 & One-stage Object Detection (often with FPN) & Accuracy (mAP - mean Average Precision), Latency, Throughput (SPS) & Data Center GPUs (NVIDIA GeForce RTX 4090/4000/2000E), Edge devices, specialized detection accelerators.&Measures the system's ability to find and localize multiple objects in images. Uses the OpenImages dataset. \\
\hline
 stable-diffusion-xl & Stable Diffusion XL (SDXL) & Text-to-Image Generation & Generative AI, Digital Content Creation	 & Diffusion Model (Latent Diffusion with U-Net) & Images Per Second, Latency (Time to generate an image) & Data Center GPUs (e.g., NVIDIA H100/H200, AMD MI300 series), powerful consumer-grade GPUs.&Represents the high-compute generative AI domain. Measures the speed of synthesizing high-resolution images from text prompts. \\
\hline
 whisper & Whisper-Large-V3 & Automatic Speech Recognition (ASR) & Speech-to-Text Services, Live Transcription	 & Encoder-Decoder Transformer & Accuracy (WER - Word Error Rate, Word Acc), Tokens Per Second & Data Center GPUs (NVIDIA GeForce RTX 4090), Edge/Client devices for real-time transcription.&A modern, high-accuracy ASR benchmark, using a Transformer architecture that handles both audio encoding and token generation. \\
\hline
%
% Inference: Mobile
%
\rowcolor{gray!20}\multicolumn{8}{|l|}{\textbf{MLPerf Inference: Mobile}} \\ \hline
 MLPerf Mobile/Edge & MobileNetV4-Conv-L & Image Classification, Object Detection & Edge/Mobile AI, low-latency on-device vision tasks. & CNN / MobileNet Family (V4) & Latency (ms), Throughput (Inferences/sec), Top-1/Top-5 Accuracy, Average Precision (AP). & Mobile SoCs, Specialized Mobile Accelerators (e.g., Apple Neural Engine, Edge TPUs, dedicated DSPs) &The largest convolutional-only variant of MobileNetV4. Optimized via Neural Architecture Search (NAS) for better latency-accuracy trade-offs on mobile and embedded hardware. \\
\hline
 MLPerf Mobile/Edge & Mobile SSD Variants & Object Detection & Edge/Mobile AI, real-time detection on resource-constrained devices. & Single Shot Detector (SSD) + Mobile Backbone & Average Precision (AP) (e.g., COCO AP), Latency (ms), FPS. & Mobile SoCs (CPU, GPU, NPU/DSP), Edge AI Accelerators& Refers to models like SSD-MobileNet V1/V2/V3 which are standard mobile benchmarks.\\
\hline
 MLPerf Edge & SSD-MobileNet & Object Detection (Small) & Edge/Mobile AI, detection for systems with tight latency/power budgets. & Single Shot Detector (SSD) + MobileNet Backbone & Average Precision (AP), Latency (ms). & Mobile SoCs (CPU, GPU, NPU/DSP), Edge AI Accelerators &A specific variant that is an original, primary benchmark for MLPerf Inference: Edge.\\
\hline
 MLPerf Mobile/Edge & MobileNet V1–V4 & Image Classification, Feature Extractor & Efficient Vision Models, low-power and low-latency inference. & CNN (V1: Depthwise Separable Convs, V2: Inverted Residuals, V3: Squeeze-and-Excitation, V4: UIB/Mobile MQA) & MACs/FLOPs, Latency (ms).& Mobile SoCs (CPU, GPU, NPU/DSP, e.g., Qualcomm Snapdragon, Apple A-series), Microcontrollers (MCUs), Edge AI Accelerators (e.g., Google Edge TPU) &A progression of architectures from Google, all focused on minimal computational cost while maintaining high accuracy, crucial for all MLPerf Edge divisions. \\
\hline
 MLPerf Mobile & MobileNet V4 & Image Classification, Object Detection & Universally Efficient AI, aiming for state-of-the-art accuracy-latency trade-offs. & Hybrid (Convolutional + Attention - Mobile MQA) & Latency (ms)& Mobile SoCs (CPU, GPU, NPU/DSP, e.g., Qualcomm Snapdragon) &The latest generation, featuring the Universal Inverted Bottleneck (UIB) and Mobile MQA. \\
\hline
 MLPerf Mobile & MOSAIC & Image Segmentation & Mobile Image Segmentation, on-device image processing. & U-Net variant with a MobileNet-style backbone. & Mean Intersection over Union (mIoU), Latency (ms).& Mobile SoCs (CPU, GPU, NPU) &A common model used for segmentation tasks in the MLPerf Mobile suite. \\
\hline
 MLPerf Mobile & MobileDETs & Object Detection & Edge/Mobile AI, high-speed detection for mobile chips. & Model Family derived from Neural Architecture Search (NAS) & Average Precision (AP), Latency (ms).& Mobile SoCs (NPU/DSP emphasized), Edge AI Accelerators &A family of detectors specifically optimized for latency on mobile SoCs. \\
\hline
 MLPerf Tiny\/Mobile & BERT-Tiny\/ DistilBERT & Natural Language Processing (NLP) Tasks (e.g., Q\&A) & Mobile\/Edge NLP, faster, smaller language understanding on local devices. & Transformer \/ Distillation Models & Latency (ms), F1 Score (SQuAD), GLUE Score. & CPUs, GPUs, Edge AI Accelerators, Mobile SoCs (optimized for low-latency) &Smaller, compressed versions of BERT achieved through knowledge distillation for resource-constrained environments. \\
\hline
 MLPerf Mobile & Mobile-BERT & Natural Language Processing (NLP) Tasks & Edge/Mobile NLP, task-agnostic BERT for resource-limited devices. & Compressed Transformer (Bottleneck structures, Knowledge Distillation) &Latency (ms), F1 Score (SQuAD), GLUE Score.& CPUs, GPUs, Edge AI Accelerators, Mobile SoCs (optimized for low-latency) &Achieves competitive results to BERT-Base with much higher speed and smaller size. \\
\hline
 MLPerf Mobile & EDSR F32B5 & Image Super-Resolution (SR) & Image Enhancement, upscaling low-resolution images for improved quality.& Enhanced Deep Super-Resolution (EDSR) Network &Latency (ms), PSNR, SSIM. & GPUs, Custom Hardware/FPGAs, specialized ISP (Image Signal Processor) components. &A common, high-quality reference model for measuring performance on image enhancement\/quality tasks.\\
\hline
 MLPerf Mobile & Stable Diffusion & Text-to-Image Generation & Generative AI, creating high-resolution images from text prompts. & Latent Diffusion Model (LDM) (U-Net, VAE, CLIP Text Encoder) & Images/Query Per Second (Throughput), Latency (Time-to-Image), FID/CLIP Scores. &High-end GPUs (e.g., NVIDIA A100/H100, RTX series), high-power Workstations and Data Center Accelerators. &A critical benchmark for measuring performance on large, complex generative workloads.\\
\hline
%
% Tiny
%
\rowcolor{gray!20}\multicolumn{8}{|l|}{\textbf{MLPerf Inference: Tiny}} \\ \hline
 MLPerf Tiny v 0.5 & Keyword Spotting Model & Audio Classification & TinyML/MCU, always-on voice assistant, device wake-word detection. &Small CNN (e.g., DS-CNN) or RNN. & Latency (ms), Energy (Joules), Area Under the ROC Curve (AUC). & Microcontrollers (MCUs) (e.g., Arm Cortex-M4\/M7), Digital Signal Processors (DSPs), Tiny Neural Network Accelerators. &Detects a specific word (e.g., "Hey Google") from a stream of audio, running on a highly constrained power budget. \\
\hline
 MLPerf Tiny  v 0.5 & Visual Wake Words (VWW) Model & Image Classification (Binary) & TinyML/MCU, low-power sensing, person detection, motion-activated cameras. & Small CNN (e.g., MobileNet V1/V2 variant). & Latency (ms), Energy (Joules), AUC. &MCUs, low-power vision processors, small-scale embedded systems. & Determines if a person is present in the image (person/not-person). Much simpler and smaller than general ImageNet classification. \\
\hline
 MLPerf Tiny  v 0.5 & Image Classification Model & Image Classification (Multi-class) & TinyML/MCU, general object recognition on ultra-low-power sensors. & Very small CNN (e.g., ResNet-8 or Micro-CNN). & Latency (ms), Energy (Joules), Top-1 Accuracy (e.g., on CIFAR-10). & MCUs with limited RAM and Flash storage. &A more complex classification task than VWW, but still constrained to a very small model size. \\
\hline
 MLPerf Tiny  v 0.5 & Anomaly Detection (AD) Model & Time Series Anomaly Detection &TinyML/MCU, industrial predictive maintenance, system health monitoring. & Small Autoencoder or similar lightweight model. & Latency (ms), Energy (Joules), AUC. & MCUs, industrial IoT sensors, devices monitoring vibration or sound. &Learns a baseline of normal sensor data (e.g., machine vibrations) and flags deviations as anomalies. \\
\hline
%
% MLPerf Client
%
\rowcolor{gray!20}\multicolumn{8}{|l|}{\textbf{MLPerf Client}} \\ \hline
 MLPerf Client & Llama 2 7B Chat & Code analysis, Content generation, Creative writing, Summarization (various lengths). &General-purpose AI, Dialogue/Chatbots, Client-side LLM inference on PCs. & Transformer, Decoder-Only, Instruction-Tuned (SFT + RLHF), 7 Billion parameters. & Time-to-First Token (TTFT), Tokens/Second (Throughput). & Client GPUs (e.g., AMD Radeon, Intel Arc), Integrated NPUs (e.g., Intel Core Ultra, AMD Ryzen AI), Data Center GPUs (e.g., NVIDIA A100/H100) for server-side inference. &A foundational model in the benchmark for measuring core client-side LLM performance.\\
\hline
 MLPerf Client & Llama 3.1 8B Instruct (8B parameters) & Generative AI workloads: Code analysis, Content generation, Creative writing, Summarization. &General-purpose AI, Instruction Following, Client-side LLM inference on PCs. & Transformer, Decoder-Only, Instruction-Tuned, 8 Billion parameters. & Time-to-First Token (TTFT) (Latency), Tokens/Second (Throughput). & Client PCs and Data Center/Cloud-based GPUs (optimized for both low-latency "Time to First Token" and high-throughput "Tokens Per Second"). &An updated and highly capable open-weight model, demonstrating improved performance and alignment over Llama 2. \\
\hline
 MLPerf Client & Phi 3.5 Mini Instruct & Reasoning (Math, Code, Logic), Long Context Query \& Summarization (up to 128K tokens).&Memory/Compute Constrained Environments, Low-Latency Applications, On-device deployment (AI PCs, mobile). & Dense Decoder-Only Transformer, Instruction-Tuned, 3.8 Billion parameters. & Time-to-First Token (TTFT) (Latency), Tokens/Second (Throughput). & Client GPUs, NPUs, and potentially high-end mobile/edge processors (optimized for on-device deployment). &A highly efficient and lightweight model optimized for speed and strong reasoning despite its small size.\\
\hline
 MLPerf Client & Phi 4 Reasoning 14B & Complex Reasoning (multi-step math, scientific, coding, planning), Generating detailed chain-of-thought traces. & Agentic applications, High-accuracy problem-solving, Applications requiring explainability. & Dense Decoder-Only Transformer, Reasoning-Focused SFT (and possible RLHF for Plus variant), 14 Billion parameters. & Time-to-First Token (TTFT) (Latency), Tokens/Second (Throughput), Accuracy on reasoning tasks. & High-performance Client PCs (Workstations) and Data Center GPUs (due to its larger size and focus on complex, token-intensive reasoning). &Included as an experimental model in the benchmark, specifically designed to emphasize logical and complex problem-solving.\\
\hline
%
% MLPerf Storage
%
\rowcolor{gray!20}\multicolumn{8}{|l|}{\textbf{MLPerf Storage}} \\ \hline
 MLPerf Storage & ResNet-50 & I/O Workload for Image Classification Training &General-purpose computer vision, low-latency image processing. & Convolutional Neural Network (CNN) & Max Supported Accelerators, Aggregate Throughput (MiB/s), Accelerator Utilization ($\ge 90\%$ required). & Data Center GPUs (NVIDIA A100/H100), Edge AI Accelerators, and high-end CPUs (widely used across all MLPerf divisions: Data Center, Edge, Tiny). &High IOPS Demand. Characterized by highly concurrent, random reads of many small data samples ($\approx 150 \text{ KB}$ each), stressing metadata and IOPS capability.\\
\hline
 MLPerf Storage & 3D U-Net & I/O Workload for Medical Image Segmentation Training & Healthcare/Radiology, Medical Image Analysis, 3D data processing. &3D U-Net (3D CNN) &Max Supported Accelerators, Aggregate Throughput (GiB/s), Accelerator Utilization ($\ge 90\%$ required). &High-end Data Center GPUs (NVIDIA A100/H100) and specialized high-throughput storage systems (MLPerf Storage benchmark focus).&High Bandwidth Demand. Characterized by concurrent random reads of very large data files ($\approx 140 \text{ MB}$ each), stressing sustained data throughput.\\
\hline
 MLPerf Storage & CosmoFlow & I/O Workload for Scientific Parameter Prediction Training & Scientific High-Performance Computing (HPC), Astrophysics. &3D Convolutional Neural Network (3D CNN) &Max Supported Accelerators, Aggregate Throughput (GiB/s), Accelerator Utilization ($\ge 70\%$ required). & Supercomputers \& HPC Clusters: Requires massive scale distributed training across hundreds or thousands of GPUs (e.g., utilizing NVIDIA H100s, Intel Gaudi, and specialized high-speed interconnects like InfiniBand). &CPU-Intensive Workload. Uses medium-sized samples ($\approx 2 \text{ MB}$), but the client-side processing is more CPU-heavy, leading to a slightly lower required accelerator utilization threshold.\\
\hline
%
% MLPERF AUTOMOTIVE
%
\rowcolor{gray!20}\multicolumn{8}{|l|}{\textbf{MLPerf Automotive}} \\ \hline
\hline
 MLPerf Automotive & SSD-ResNet50 & 2D Object Recognition and Segmentation &
ADAS / Collision Avoidance, Lane Departure & Single Shot Detector (SSD) with ResNet-50 Backbone &
Latency, Throughput, $mAP$ (Accuracy) &Edge AI Accelerators, Embedded GPUs, and Automotive System-on-Chips (SoCs).& Baseline benchmark for camera-based detection on high-res (8MP) images. Used in v0.5. \\
\hline
 MLPerf Automotive & BEVFormer-Tiny & Camera-based 3D Object Detection & Autonomous Driving (L2+ to L4), Environmental Perception & Bird's Eye View (BEV) Transformer-based Network &
Latency, Throughput, $\text{mAP}$ (Accuracy) & High-compute Automotive SoCs, next-generation AI accelerators (specifically targeting transformer and multi-sensor fusion capabilities). &Represents state-of-the-art camera-only 3D perception. Used in MLPerf Auto v0.5. \\
\hline
 MLPerf Automotive & DeepLabV3Plus / PointPainting & Semantic Segmentation (as a component of 3D Detection) & Lidar-Camera Sensor Fusion, 3D Perception & DeepLabV3+ (for Segmentation) + PointPillars (for 3D Detection) & Latency ($p99.9$ percentile), Throughput, Accuracy) &Safety-critical Automotive SoCs, purpose-built AI processors for ADAS/AV, often requiring high-reliability and low-latency performance.&
DeepLabV3+ is the 2D segmentation part of the PointPainting sensor fusion pipeline. Used in MLPerf Inference v5.0 Automotive. \\
\hline
%
% MLPERF Training:HPC
%
\rowcolor{gray!20}\multicolumn{8}{|l|}{\textbf{MLPerf Training:HPC}} \\ \hline
\hline
MLPerf Training:HPC & CosmoFlow &Prediction of Cosmological Parameters ($\Omega_m, \sigma_8, n_s, H$) &
Astrophysics, Cosmology, Scientific Simulation Parameter Prediction & 3D Convolutional Neural Network (3D CNN)&
Time-to-Train (Total time to reach a target quality metric), Aggregate Throughput (Models trained per unit of time in weak scaling). &Supercomputers \& Large Clusters (e.g., NVIDIA Selene, Perlmutter, Fugaku), utilizing thousands of interconnected High-Performance GPUs (e.g., NVIDIA A100/H100) and high-speed parallel file systems.&Trained on 3D volumetric data (dark matter distributions) from N-body simulations. The large, volumetric data introduces significant I/O challenges and stresses high-bandwidth interconnects and storage. \\
\hline
MLPerf Training:HPC & DeepCAM & Semantic Segmentation of Extreme Weather Events (e.g., atmospheric rivers, tropical cyclones) &
Climate Science, Weather Forecasting, Earth System Modeling & Convolutional Encoder-Decoder (U-Net variant) & Time-to-Train (Total time to reach a target quality metric), Aggregate Throughput (Models trained per unit of time in weak scaling). & Supercomputers \& Large Clusters, demanding high I/O bandwidth to handle the massive 8.8 TB climate datasets and requiring excellent strong-scaling performance. This benchmark was running on NVIDIA V100/A100.&Trained on massive, high-resolution, multi-channel images (e.g., $768 \times 1152$ pixels with $16$ channels). Features high computational intensity and large memory footprint per sample. \\
\hline
MLPerf Training:HPC & OpenCatalyst & Prediction of energy and forces for molecular systems (AI for materials science) & Catalyst Discovery, Computational Chemistry, Materials Science, Energy Storage &
Graph Neural Network (GNN), specifically DimeNet++ & Time-to-Train (Total time to reach a target quality metric), Aggregate Throughput (Models trained per unit of time in weak scaling). & Supercomputers \& Large Clusters, typically emphasizing the performance of GNNs, which stress different aspects of the system, like memory access patterns and graph-specific operations. This benchmark was running on NVIDIA V100/A100.& Predicts quantum mechanical properties of catalyst systems. Stresses complex data structures (graphs) and large-scale parallel processing. Uses the massive OC20 dataset. \\
\hline
%
% SCIENCE
%
\rowcolor{gray!20}\multicolumn{8}{|l|}{\textbf{MLCommons Science}} \\ \hline
\hline
MLCommons Science & Cloud Mask &  image processing / segmentation & Earth Observation, Segmentation model for the pixel classification in satellite images & U-Net deep neural network  &training and inference timing and scalability on the training across a number of GPUs;runtime of training and inference. & HPC Clusters \& High-Performance GPUs (e.g., NVIDIA A100/V100) running distributed training frameworks like PyTorch or TensorFlow, often benchmarked for large-scale data I/O. & Focuses on identifying and isolating cloud cover in high-resolution satellite imagery for subsequent analysis. \\ 
\hline
MLCommons Science & STEMDL & A universal classifier for the space group of solid-state materials.
 & Scientific Machine Learning (General benchmark suite) &CNN: ResNet, VGG, DenseNet &  top1 accuracy and F1 score (Macro)& HPC Systems of all sizes, used for general performance comparison across different hardware architectures and scaling tests. This benchmrak was running NVIDIA A100/V100.& The goals of this benchmark are to: (1) explore the suitability of machine learning algorithms in the advanced analysis of Convergent beam electron diffraction (CBED) and (2) produce a machine learning algorithm capable of overcoming intrinsic difficulties posed by scientific datasets. \\
\hline
MLCommons Science & CANDLE UNO & Cancer Drug Response Prediction &Life Sciences / Personalized Medicine   & Neural Networks(MLP)& TTT, Prediction Accuracy & HPC Systems (e.g., Summit, Polaris) and Cloud Environments, stressing both compute performance and workflow management for parameter sweep tasks.  This benchmark was running NVIDIA A100.& Benchmarks deep learning models for predicting the response of various cancer cell lines to different therapeutic compounds. \\
\hline
MLCommons Science & Earthquake & TEvolOp Earthquake Forecasting Model &Earthquake Science & Neural Networks(MLP)- recurrent neural networks and transformers &Nash Sutcliffe efficiency &HPC \& Big Data Systems, requiring efficient handling of large, continuous time-series datasets and high-throughput data processing. This benchmark was running on NVIDIA V100.& Benchmarks deep learning models for predicting the response of various cancer cell lines to different therapeutic compounds. \\
\hline 
%
% AlgoPerf
%
\rowcolor{gray!20}\multicolumn{7}{|l|}{\textbf{MLCommons AlgoPerf}} \\ \hline
\hline
AlgoPerf &  Criteo 1TB & Click-Through Rate (CTR) Prediction & Large-scale Recommender Systems, Digital Advertising & DLRM-Small (Deep Learning Recommendation Model) &Time-to-Result (Time to reach a target AUC) & Datacenter CPUs/GPUs with high memory bandwidth (HBM) due to massive embedding tables, and highly optimized network I/O. & Stresses memory access and sparse feature embedding computations due to the large, sparse Criteo 1TB dataset. Represents a common commercial workload. \\
\hline 
AlgoPerf &  FastMRI & k-space MRI Reconstruction & Medical Imaging, Healthcare Diagnostics & U-Net (Convolutional Encoder-Decoder) &Time-to-Result (Time to reach a target PSNR / SSIM)&High-Performance GPUs and dedicated AI accelerators, as the model must run with high accuracy and low latency for clinical use.&Focuses on accelerating the image formation process from raw MRI data. U-Net is a standard model for semantic segmentation and image-to-image translation tasks.\\
\hline 
AlgoPerf &  ImageNet & Image Classification & General-purpose Computer Vision & ResNet-50 and Vision Transformer (ViT) variants & Time-to-Result (Time to reach a target Top-1 Accuracy & General-Purpose GPUs (Training/Inference), Edge Devices, and Mobile SoCs, as it is a widely-used test across all compute scales. &The quintessential computer vision workload. Includes two major architecture types (CNN and Transformer) to test algorithm generalizability.\\
\hline
AlgoPerf & LibriSpeech  &Speech Recognition / ASR (Automatic Speech Recognition)& Voice Assistants, Transcription Services & Conformer and DeepSpeech variants & Time-to-Result (Time to reach a target Word Error Rate (WER)) & Datacenter/Cloud GPUs (for large-scale ASR), Edge/Mobile Processors (for on-device assistants). &Tests algorithms on sequential data. Conformer is a hybrid CNN/Transformer architecture common in modern ASR.\\
\hline
AlgoPerf & OGBG  &Graph Property Prediction & Scientific Machine Learning, Drug Discovery, Social Networks & GNN (Graph Neural Network) & Time-to-Result (Time to reach a target ROC-AUC) &Datacenter CPUs/GPUs with high-speed interconnects due to the irregular, sparse nature of graph-structured data. &Uses the Open Graph Benchmark (OGB) dataset. This workload stresses algorithms in domains that rely on non-Euclidean data structures.\\
\hline
AlgoPerf & WMT  & Machine Translation (En-De) &Natural Language Processing (NLP), Global Communication & Transformer (Base Architecture) & Time-to-Result (Time to reach a target BLEU Score) &Datacenter CPUs/GPUs with specialized Tensor Cores for efficient processing of the Transformer's self-attention mechanism.& A standard, large-scale sequence-to-sequence task, famous for being the original domain of the Transformer architecture.\\
\hline
%
% AILuminate
%
\rowcolor{gray!20}\multicolumn{7}{|l|}{\textbf{MLCommons AILuminate}} \\ \hline
AILuminate Safety v1.0 & System Under Test (SUT) (Any LLM-based general-purpose chat system)  & Assess Baseline AI Safety and Reliability &Pre-deployment Validation, Regulatory Compliance, Vendor Comparison & LLMs and AI Chat Systems (Text-to-Text), potentially with guardrails/filters &Overall Safety Grade (5-tier scale: Poor to Excellent), Violation Rate (\% of unsafe responses), Per-Hazard Performance &The AI System itself (typically hosted in a Datacenter/Cloud) is the system under test (SUT). The evaluation is performed by a separate, specialized Safety Evaluator Model (often a tuned LLM ensemble). &Assesses safety against 12 Hazard Categories (e.g., Violent Crimes, Hate, Suicide \& Self-Harm). Uses a tuned ensemble of safety evaluator models for grading. Focuses on single-turn, content-only hazards.\\
\hline
AILuminate Jailbreak Benchmark v0.5 & System Under Test (SUT) (Any LLM-based general-purpose chat system)  & Quantify Resilience to Adversarial "Jailbreak" Attacks & AI Security, Robustness Testing, Defense Mechanism Comparison & LLMs (Text-to-Text) and Vision-Language Models (VLMs) (Text+Image-to-Text) & Resilience Gap (Drop in safety performance from baseline to under-attack), Jailbreak Success Rate &The AI System (SUT) is tested in a Datacenter/Cloud environment. The benchmark focuses on the input (adversarial prompts) and the system's subsequent failure rate under attack conditions. & v0.5 is an initial release establishing the framework. It specifically measures the degradation of safety when a system is subjected to prompts designed to bypass its safety filters ("jailbreaks").\\
\hline
\end{longtable}
\end{landscape} 
\twocolumn

}
\clearpage


%
% AILuminate
%


\onecolumn
\begin{landscape}

{\tiny
\begin{longtable}
{|p{
0.1\textwidth}|p{
0.1\textwidth}|p{
0.1\textwidth}|p{
0.2\textwidth}|p{
0.2\textwidth}|p{
0.2\textwidth}|p{
0.2\textwidth}|}
\caption{Large Language Model Benchmark Details}
\label{tab:llm_benchmarks_long} \\
\hline
\rowcolor{blue!30}
\textbf{Benchmark Name} & \textbf{Model} & \textbf{Task} & \textbf{Application Domain / Use Case} & \textbf{Model Type / Architecture} & \textbf{Metrics / KPIs} & \textbf{Notes / Description} \\
\hline
\endfirsthead

\rowcolor{blue!30}\multicolumn{7}{c}%
{{\bfseries \tablename\ \thetable{} -- Continued from previous page}} \\
\toprule
\textbf{Benchmark Name} &
\textbf{Model} &
\textbf{Task} &
\textbf{Application Domain / Use Case} &
\textbf{Model Type / Architecture} &
\textbf{Metrics / KPIs} &
\textbf{Notes / Description} \\
\midrule
\endhead

\midrule
\multicolumn{7}{r}{{\footnotesize Continued on next page}} \\
\endfoot

\bottomrule
\endlastfoot
\hline
\rowcolor{gray!20}\multicolumn{7}{|l|}{\textbf{Commercial/Proprietary LLMs (API/Systems)}} \\
\hline
LLM Inference & Claude 3.5 Haiku 20241022 & Generative AI & General Purpose, Light Reasoning & Large Transformer (Proprietary) & TTFT, TPOT, Throughput, MMLU (Quality) & A faster, smaller version in the Claude 3.5 family. \\
\hline
LLM Inference & Claude 3.5 Sonnet 20241022 & Generative AI & Complex Reasoning, Data Processing & Large Transformer (Proprietary) & TTFT, TPOT, Throughput, MMLU (Quality) & Mid-tier model focusing on balance of speed and intelligence. \\
\hline
LLM Inference & Mistral Large 2402 Moderated & Generative AI & Enterprise Chatbots, Content Moderation & MoE/Dense Transformer (Proprietary) & TTFT, TPOT, Throughput, Safety Index & Flagged as moderated; emphasis on safety and reliable output. \\
\hline
LLM Inference & Amazon Nova Lite v1.0 & Generative AI & AWS Services, Embedded Use Cases & Large Transformer (Proprietary) & Latency, Throughput, Cost/Token & Lightweight, cloud-optimized model. \\
\hline
LLM Inference & Gemini 1.5 Pro (API, with option) & Generative AI / Multimodal & Long Context, Multi-Source Reasoning & MoE/Dense Transformer (Proprietary, Multimodal) & TTFT, Throughput, Latency, RAG/Context Recall & Known for its massive context window. \\
\hline
LLM Inference & Gemini 2.0 Flash 001 & Generative AI / Multimodal & High-Speed Chat, Real-time Tasks & Dense Transformer (Proprietary, Multimodal) & p99 Latency, Throughput & Focuses on speed and efficiency for low-latency needs. \\
\hline
LLM Inference & Gemini 2.0 Flash Lite & Generative AI & Edge/Client-Side Inference & Dense Transformer (Proprietary, Small) & Energy Efficiency, Latency & Highly optimized for resource-constrained environments. \\
\hline
LLM Inference & GPT-4o & Generative AI / Multimodal & Real-time Conversation, Vision Integration & Dense Transformer (Proprietary, Multimodal) & TTFT, TPOT, Low-Latency Response & All-in-one model for low-latency multimodal interactions. \\
\hline
LLM Inference & GPT-4o mini & Generative AI & Quick, Cost-Effective Tasks & Dense Transformer (Proprietary, Small) & Cost/Token, Throughput & Optimized for efficiency and scaling simple tasks. \\
\hline
LLM Inference & Minustral 8B 24.10 (API) & Generative AI & General Text Generation & MoE/Dense Transformer (Proprietary) & Latency, Throughput & Represents a competitive, smaller model in a commercial API. \\
\hline
\rowcolor{gray!20}\multicolumn{7}{|l|}{\textbf{Open-Source/Bare Models (Used for Training or Deployment)}} \\
\hline
LLM Inference & Minustral 8B 24.10 Moderation & Generative AI & General Text Generation, Safety Research & MoE/Dense Transformer (Open-weights) & Latency, Safety Compliance & Open-weight version with a focus on safety. \\
\hline
LLM Inference & Gemma 2 9b & Generative AI & Fine-tuning, Edge Deployment & Dense Transformer (Open-weights) & Perplexity, MMLU, Throughput & Smaller model from the Gemma family, good for fine-tuning. \\
\hline
LLM Inference & Phi 3.5 MoE Instruct & Generative AI & Instruction Following, Small Scale Reasoning & MoE (Open-weights, Small) & MMLU, HumanEval (Code) & Instruction-tuned, likely using a small Mixture-of-Experts. \\
\hline
LLM Inference & Phi 4 & Generative AI & Research, Prototyping & Dense Transformer (Open-weights, Small) & Perplexity, BLEU (Generation) & Successor in the Phi family, typically very small. \\
\hline
LLM Inference & Athene V2 Chat Hf & Generative AI & Open Chatbot Deployment & Dense Transformer (Open-weights, Fine-tuned) & TTFT, TPOT, Chat Metrics & An instruction-tuned model from the Hugging Face ecosystem. \\
\hline
LLM Inference & Aya Expanse 8B Hf & Generative AI & Multilingual Tasks, Text Translation & Dense Transformer (Open-weights) & BLEU (Translation), Accuracy & Focused on broad language coverage. \\
\hline
LLM Inference & Cohere C4Ai Command A 03 2025 Hf & Generative AI & Enterprise RAG, Instruction Following & Dense Transformer (Open-weights) & Contextual Recall, RAG Latency & Cohere model variant used in the Hugging Face ecosystem. \\
\hline
LLM Inference & Llama 3.1 405B Instruct & Generative AI & State-of-the-Art Reasoning, Long Context & Dense Transformer (Open-weights) & TTFT, Throughput, MMLU & An extremely large, cutting-edge open-weight model (used in MLPerf). \\
\hline
LLM Inference & Llama 3.1 8b Instruct FP8 & Generative AI & Edge/Quantized Deployment & Dense Transformer (Quantized) & Inference Accuracy, Memory Footprint & Highly optimized for efficient computation using 8-bit precision. \\
\hline
LLM Inference & Llama 3 1 Tulu 3 8B Hf & Generative AI & General Chat, Fine-tuning Research & Dense Transformer (Open-weights, Fine-tuned) & Alpaca Eval, Human Preference & A variant of Llama tuned for instruction following. \\
\hline
LLM Inference & Mistralai Mistral Large 2402 & Generative AI & Complex Reasoning, RAG & MoE/Dense Transformer (Open-weights) & TTFT, TPOT, MMLU & Open-weight version of Mistral's flagship model. \\
\hline
LLM Inference & Olmo 2 0325 32b Instruct & Generative AI & Research, Reproducible AI & Dense Transformer (Open-weights) & Perplexity, Training Speed & High-parameter model focused on openness and research. \\
\hline
LLM Inference & Olmo 2 1124 13B Instruct Hf & Generative AI & Instruction Following, General Chat & Dense Transformer (Open-weights) & TTFT, Throughput & Smaller, instruction-tuned version of the Olmo family. \\
\hline
LLM Inference & Phi 3.5 Mini Instruct & Generative AI & Mobile/Edge Inference, Simple Tasks & Dense Transformer (Open-weights, Small) & Latency, MMLU & Ultra-small model optimized for fast responses. \\
\hline
LLM Inference & Qwen1 5 110B Chat Hf & Generative AI & Multi-Language Chat, High Accuracy & Dense Transformer (Open-weights) & C-Eval, MMLU, Throughput & High-parameter model known for strong Chinese/general performance. \\
\hline
LLM Inference & Yi 1 5 34B Chat Hf & Generative AI & General Purpose, Instruction Following & Dense Transformer (Open-weights) & MMLU, C-Eval, Latency & Mid-to-large size model focusing on quality chat performance. \\
\hline
LLM Inference & Ai21Labs Ai21 Jamba Large 1.5 Azure & Generative AI & Cloud Deployment, Enterprise Apps & Hybrid MoE/Dense Transformer & Throughput, Latency & A large model known for its hybrid architecture, deployed via Azure. \\
\hline
LLM Inference & Google Gemma 3 27B It Hf Nebius & Generative AI & Cloud Deployment, Fine-tuning & Dense Transformer (Open-weights, Fine-tuned) & TTFT, TPOT, Cloud Efficiency & Gemma model deployed on the Nebius cloud platform. \\
\hline
LLM Inference & Llama 3.3 70B Instruct Turbo Together & Generative AI & Fast, High-Quality Instruction Following & Dense Transformer (Open-weights) & Latency, Throughput, Cost & A large model optimized for speed via the Together API. \\
\hline
LLM Inference & Mistral Large 24.11 & Generative AI & Enterprise AI, High Performance & MoE/Dense Transformer (Open-weights) & Throughput, MMLU, Reasoning & A very recent high-performance model. \\
\hline
LLM Inference & Qwq 32B Hf & Generative AI & General Purpose, Instruction Following & Dense Transformer (Open-weights) & Latency, Throughput & A mid-sized model in the open-weight ecosystem. \\
\hline
LLM Inference & OLMo 7b 0724 Instruct & Generative AI & Research, Instruction Following & Dense Transformer (Open-weights) & Perplexity, Speed & Smaller, instruction-tuned model for general tasks. \\

\end{longtable}
}
    
\end{landscape}
\twocolumn



\clearpage

%\begin{table*}[htbp]
\centering
\caption{Machine Learning Benchmarks Across Various Domains \TODO{citations missing. See \url{https://mlcommons-science.github.io/benchmark/}}}
\label{tab:ml_benchmarks}
\resizebox{\textwidth}{!}{
\begin{tabular}{|l|l|l|l|l|}
\hline
\rowcolor{blue!30}
\textbf{Benchmark} & \textbf{Domain} & \textbf{ML motif} & \textbf{Computing motif} \\
\hline
\hline
\YES jet classification \citep{duarte2022fastml} & HEP/NP & Classification & Embedded \\
\hline
\YES qubit readout \citep{diguglielmo2025endtoendworkflowmachinelearningbased} & QIS & Classification, Time-series & Embedded \\
\hline
\YES quench detection \citep{quench2024} & HEP/NP & Classification, Time-series & Embedded, Edge Inference \\
\hline
\YES BraggNN \citep{liu2021braggnnfastxraybragg} & Mat Sci & Regression & Embedded \\
\hline
\YES smart pixels \citep{parpillon2024smartpixelsinpixelai} & HEP/NP & Classification & Embedded \\
\hline
\YES Calo challenge \citep{krause2024calochallenge2022communitychallenge} \citep{calo-surrogate} & HEP/NP & Generative & HPC training, Inference \\
\hline
\YES Cloud Masking \citep{las-mlcommons-science} & Bio/Env Sci & Semantic Segmentation, Computer Vision & Inference \\
\hline
\YES CosmoFlow \citep{farrell2021mlperfhpcholisticbenchmark} & HEP/NP & Computer Vision, Inverse & HPC training, Inference \\
\hline
\YES Deep CAM \citep{farrell2021mlperfhpcholisticbenchmark} & Bio/Env Sci & Semantic Segmentation, Computer Vision & HPC training, Inference \\
\hline
\YES Earthquake forecasting \citep{las-mlcommons-science} & Bio/Env Sci & Regression, Time-series & Inference \\
\hline
\YES STEM DL \citep{las-mlcommons-science} & Mat Sci & Classification, Computer Vision & Inference \\
\hline
\YES Uno \citep{las-mlcommons-science} & Medical & Regression & Inference \\
\hline
\YES PDE Bench \citep{takamoto2024pdebenchextensivebenchmarkscientific} & Math & PINNs & Inference \\
\hline
\YES The Well \citep{neurips2024_4f9a5acd} & Math & PINNs & Inference \\
\hline
\YES HDR Anomaly Challenge GW \citep{campolongo2025buildingmachinelearningchallenges} & HEP/NP & Anomaly Detection & Inference \\
\hline
\YES HDR Anomaly Challenge Butterfly \citep{campolongo2025buildingmachinelearningchallenges2} & Bio/Env Sci & Anomaly Detection & Inference \\
\hline
\YES HDR Anomaly Challenge Sea level \citep{campolongo2025buildingmachinelearningchallenges3} & Bio/Env Sci & Anomaly Detection & Inference \\
\hline
\YES MHD regression \citep{Wei_2024} & Fusion & Regression & Embedded \\
\hline
\NO PtychoNN \citep{sbifair-ptycho} & Mat Sci & Regression & Embedded, Edge Inference \\
\hline
\NO ECON autoencoder [citation pending] & HEP/NP & Compression & Embedded \\
\hline
\NO supernova pointing \citep{UBOLDI2022166371} & HEP/NP & Regression & Edge Inference \\
\hline
\NO TPBench \citep{chung2025theoreticalphysicsbenchmarktpbench} & HEP/NP & LLM & HPC training, Inference \\
\hline
\NO Diffuse Multiple Scattering [citation pending] & Mat Sci & Classification, Computer Vision & Inference \\
\hline
\NO Electron Microscopy Denoising [citation pending] & Mat Sci & Computer Vision, Regression & Inference \\
\hline
\NO EAIRA Astro \citep{cappello2025eairaestablishingmethodologyevaluating} & HEP/NP & LLM & Inference \\
\hline
\NO EAIRA Molecule \citep{cappello2025eairaestablishingmethodologyevaluating} & Mat Sci & LLM & Inference \\
\hline
\end{tabular}
}
\YES = included, \NO = not included in github at \citep{www-las-mlcommons-benchmark-coolection}. Benchmarks that are not currently present in github are in the process of being added. 
\end{table*}

%ML Benchmarks LaTeX Table \ref{tab:ml_benchmarks}


%\newlength{\ratingswidth}
\setlength{\ratingswidth}{0.11\textwidth}

\newlength{\imageratingswidth}
\setlength{\imageratingswidth}{0.06\textwidth}



\begin{landscape}
{\footnotesize
\begin{longtable}{
|p{0.08\textwidth}
|p{0.2\textwidth}
|p{0.2\textwidth}
|p{0.2\textwidth}
|p{0.3\textwidth}
|p{0.11\textwidth}
|}
\caption{Ontology Table for Selected AI Science Benchmarks} \label{tab:ontology} \\
\hline
\textbf{Ratings} & \textbf{Name} & \textbf{Domain} & \textbf{Models} & \textbf{Metrics} & \textbf{Citation} \\ \hline
\endfirsthead
\hline
\textbf{Ratings} & \textbf{Name} & \textbf{Domain} & \textbf{Models} & \textbf{Metrics} & \textbf{Citation}  \\ \hline \\ \hline
\endhead
\hline
\multicolumn{12}{r}{Continued on next page} \\
\endfoot
\hline
\endlastfoot
\includegraphics[width=\imageratingswidth]{climatelearn_-_weather_forcasting_radar.pdf} & ClimateLearn - Weather Forcasting & Climate \ & RMSE, Anomaly correlation & Global weather prediction (3-5 days) & \textbf{5.00} \\ \hline
\includegraphics[width=\imageratingswidth]{climatelearn_-_downscaling_radar.pdf} & ClimateLearn - Downscaling & Climate \ & RMSE, Anomaly correlation & Global weather prediction (3-5 days) & \textbf{5.00} \\ \hline
\includegraphics[width=\imageratingswidth]{climatelearn_-_climate_projection_radar.pdf} & ClimateLearn - Climate Projection & Climate \ & RMSE, Anomaly correlation & Global weather prediction (3-5 days) & \textbf{5.00} \\ \hline
\includegraphics[width=\imageratingswidth]{mlcommons_science_-_cloudmask_radar.pdf} & MLCommons Science - CloudMask & Climate \ & MAE, Accuracy, Speedup vs simulation & Inference accuracy, simulation speed-up, generalization & \textbf{5.00} \\ \hline
\includegraphics[width=\imageratingswidth]{mlcommons_science_-_earthquake_radar.pdf} & MLCommons Science - Earthquake & Climate \ & MAE, Accuracy, Speedup vs simulation & Inference accuracy, simulation speed-up, generalization & \textbf{5.00} \\ \hline
\includegraphics[width=\imageratingswidth]{mlcommons_science_-_candle_uno_radar.pdf} & MLCommons Science - Candle UNO & Biology \ & MAE, Accuracy, Speedup vs simulation & Inference accuracy, simulation speed-up, generalization & \textbf{5.00} \\ \hline
\includegraphics[width=\imageratingswidth]{mlcommons_science_-_stemdl_radar.pdf} & MLCommons Science - STEMDL & Materials Science & CNN, GNN, Transformer & MAE, Accuracy, Speedup vs simulation & \cite{10.1007/978-3-031-23220-6_4} \\ \hline \\ \hline
\includegraphics[width=\imageratingswidth]{arc-challenge_advanced_reasoning_challenge_radar.pdf} & ARC-Challenge (Advanced Reasoning Challenge) & Computational Science \ & Commonsense and scientific reasoning & Generalization & GPT-4, Claude \\ \hline
\includegraphics[width=\imageratingswidth]{molgen_radar.pdf} & MOLGEN & Chemistry & MolGen & Validity\%, Novelty\%, QED, Docking score, penalized logP & \cite{fang2024domainagnosticmoleculargenerationchemical} \\ \hline \\ \hline
\includegraphics[width=\imageratingswidth]{open_graph_benchmark_ogb_-_biology_radar.pdf} & Open Graph Benchmark (OGB) - Biology & Biology \ & Accuracy, ROC-AUC & Scalability and generalization in graph ML for biology & \textbf{4.83} \\ \hline
\includegraphics[width=\imageratingswidth]{llms_for_crop_science_radar.pdf} & LLMs for Crop Science & Climate \ & Scientific knowledge, crop reasoning & Generalization & GPT-3.5, GPT-4, Claude-3-opus, Qwen-max, LLama3-8B, InternLM2-7B, Qwen1.5-7B \\ \hline
\includegraphics[width=\imageratingswidth]{scicode_radar.pdf} & SciCode & Computational Science \ & Solve rate (\%) & Program synthesis, scientific computing & \textbf{4.50} \\ \hline
\includegraphics[width=\imageratingswidth]{calochallenge__radar.pdf} & CaloChallenge 2022 & High Energy Physics & VAE variants, GAN variants, Normalizing flows, Diffusion models & Histogram similarity, Classifier AUC, Generation latency & \cite{krause2024calochallenge2022communitychallenge} \\ \hline \\ \hline
\includegraphics[width=\imageratingswidth]{pdebench_radar.pdf} & PDEBench & Computational Science \ & Time-dependent PDE modeling; physical accuracy & Regression & FNO, U-Net, PINN, Gradient-Based inverse methods \\ \hline
\includegraphics[width=\imageratingswidth]{urban_data_layer_udl_-_pm_concentration_prediction_radar.pdf} & Urban Data Layer (UDL) - PM2.5 Concentration Prediction & Climate \ & Task-specific accuracy or RMSE & Multi-modal urban inference, standardization & \textbf{4.50} \\ \hline
\includegraphics[width=\imageratingswidth]{urban_data_layer_udl_-_built-up_area_classification_radar.pdf} & Urban Data Layer (UDL) - Built-up Area Classification & Climate \ & Task-specific accuracy or RMSE & Multi-modal urban inference, standardization & \textbf{4.50} \\ \hline
\includegraphics[width=\imageratingswidth]{urban_data_layer_udl_-_administrative_boundaries_identification_radar.pdf} & Urban Data Layer (UDL) - Administrative Boundaries Identification & Climate \ & Task-specific accuracy or RMSE & Multi-modal urban inference, standardization & \textbf{4.50} \\ \hline
\includegraphics[width=\imageratingswidth]{urban_data_layer_udl_-_el_nino_anomaly_detection_radar.pdf} & Urban Data Layer (UDL) - El Nino Anomaly Detection & Climate \ & Task-specific accuracy or RMSE & Multi-modal urban inference, standardization & \textbf{4.50} \\ \hline
\includegraphics[width=\imageratingswidth]{spiqa_llm_radar.pdf} & SPIQA (LLM) & Computational Science \ & Accuracy, F1 score & Visual reasoning, scientific figure understanding & 4.42 \\ \hline
\includegraphics[width=\imageratingswidth]{mlcommons_medical_ai_-_pancreas_segmentation_dfci_radar.pdf} & MLCommons Medical AI - Pancreas Segmentation (DFCI) & Biology \ & ROC AUC, Accuracy, Fairness metrics & Clinical accuracy, fairness, generalizability, privacy compliance & 4.33 \\ \hline
\includegraphics[width=\imageratingswidth]{mlcommons_medical_ai_-_brain_tumor_segmentation_brats_radar.pdf} & MLCommons Medical AI - Brain Tumor Segmentation (BraTS) & Biology \ & ROC AUC, Accuracy, Fairness metrics & Clinical accuracy, fairness, generalizability, privacy compliance & 4.33 \\ \hline
\includegraphics[width=\imageratingswidth]{mlcommons_medical_ai_-__surgical_workflow_phase_recognition_surgmlcube_radar.pdf} & MLCommons Medical AI -  Surgical Workflow Phase Recognition (SurgMLCube) & Biology \ & ROC AUC, Accuracy, Fairness metrics & Clinical accuracy, fairness, generalizability, privacy compliance & 4.33 \\ \hline
\includegraphics[width=\imageratingswidth]{seafloorai_radar.pdf} & SeafloorAI & Climate \ & Segmentation pixel accuracy, QA accuracy & Geospatial understanding, multimodal reasoning & 4.33 \\ \hline
\includegraphics[width=\imageratingswidth]{seafloorgenai_radar.pdf} & SeafloorGenAI & Climate \ & Geospatial understanding, multimodal reasoning & Generalization & SegFormer, ViLT-style multimodal models \\ \hline
\includegraphics[width=\imageratingswidth]{gess_-_track_pileup_radar.pdf} & GeSS - Track Pileup & High Energy Physics & GCN, EGNN, DimeNet++ & Accuracy, RMSE, OOD robustness delta & \cite{neurips2024_a8063075} \\ \hline \\ \hline
\includegraphics[width=\imageratingswidth]{gess_-_track_signal_radar.pdf} & GeSS - Track Signal & High Energy Physics & GCN, EGNN, DimeNet++ & Accuracy, RMSE, OOD robustness delta & \cite{neurips2024_a8063075} \\ \hline \\ \hline
\includegraphics[width=\imageratingswidth]{gess_-_drugood_radar.pdf} & GeSS - DrugOOD & Biology \ & Accuracy, RMSE, OOD robustness delta & OOD performance in scientific settings & 4.33 \\ \hline
\includegraphics[width=\imageratingswidth]{gess_-_qmof_radar.pdf} & GeSS - QMOF & Materials Science & GCN, EGNN, DimeNet++ & Accuracy, RMSE, OOD robustness delta & \cite{neurips2024_a8063075} \\ \hline \\ \hline
\includegraphics[width=\imageratingswidth]{ocp_open_catalyst_project_radar.pdf} & OCP (Open Catalyst Project) & Chemistry, Materials Science & CGCNN, SchNet, DimeNet++, GemNet-OC & MAE (energy), MAE (force) & \cite{chanussot2021oc20,tran2023oc22,doi:10.1021/acscatal.0c04525,tran2023b} \\ \hline \\ \hline
\includegraphics[width=\imageratingswidth]{jet_classification_radar.pdf} & Jet Classification & High Energy Physics & Keras DNN, QKeras quantized DNN & Accuracy, AUC & \cite{duarte2022fastml} \\ \hline \\ \hline
\includegraphics[width=\imageratingswidth]{irregular_sensor_data_compression_radar.pdf} & Irregular Sensor Data Compression & High Energy Physics & Autoencoder, Quantized autoencoder & MSE, Compression ratio & \cite{duarte2022fastmlsciencebenchmarksaccelerating2} \\ \hline \\ \hline
\includegraphics[width=\imageratingswidth]{mlperf_hpc_-_cosmoflow_radar.pdf} & MLPerf HPC - Cosmoflow & High Energy Physics & CosmoFlow, DeepCAM, OpenCatalyst & Training time, Accuracy, GPU utilization & \cite{farrell2021mlperfhpcholisticbenchmark} \\ \hline \\ \hline
\includegraphics[width=\imageratingswidth]{mlperf_hpc_-_deepcam_radar.pdf} & MLPerf HPC - DeepCAM & Climate \ & Training time, Accuracy, GPU utilization & Scaling efficiency, training time, model accuracy on HPC & 4.17 \\ \hline
\includegraphics[width=\imageratingswidth]{mlperf_hpc_-_open_catalyst_project_dimenet__radar.pdf} & MLPerf HPC - Open Catalyst Project DimeNet++ & Chemistry & DeepCAM & Training time, Accuracy, GPU utilization & \cite{farrell2021mlperfhpcholisticbenchmark} \\ \hline \\ \hline
\includegraphics[width=\imageratingswidth]{mlperf_hpc_-_openfold_radar.pdf} & MLPerf HPC - OpenFold & Biology \ & Training time, Accuracy, GPU utilization & Scaling efficiency, training time, model accuracy on HPC & 4.17 \\ \hline
\includegraphics[width=\imageratingswidth]{hdr_ml_anomaly_challenge_-_gravitational_waves_radar.pdf} & HDR ML Anomaly Challenge - Gravitational Waves & High Energy Physics & Deep latent CNNs, Autoencoders & ROC-AUC, Precision/Recall & \cite{campolongo2025buildingmachinelearningchallenges} \\ \hline \\ \hline
\includegraphics[width=\imageratingswidth]{supercond_-_property_prediction_radar.pdf} & SuperCon3D - Property Prediction & Materials Science & SODNet, DiffCSP-SC & MAE (Tc), Validity of generated structures & \cite{neurips2024_c4e3b55e} \\ \hline \\ \hline
\includegraphics[width=\imageratingswidth]{supercond_-__inverse_crystal_structure_generation_radar.pdf} & SuperCon3D -  Inverse Crystal Structure Generation & Materials Science & SODNet, DiffCSP-SC & MAE (Tc), Validity of generated structures & \cite{neurips2024_c4e3b55e} \\ \hline \\ \hline
\includegraphics[width=\imageratingswidth]{baisbench_biological_ai_scientist_benchmark_-_question_answering_radar.pdf} & BaisBench (Biological AI Scientist Benchmark) - Question Answering & Biology \ & Autonomous biological research capabilities & Generalization & LLM-based AI scientist agents \\ \hline
\includegraphics[width=\imageratingswidth]{baisbench_biological_ai_scientist_benchmark_-_cell_type_annotation_radar.pdf} & BaisBench (Biological AI Scientist Benchmark) - Cell Type Annotation & Biology \ & Annotation accuracy, QA accuracy & Autonomous biological research capabilities & 4.00 \\ \hline
\includegraphics[width=\imageratingswidth]{the_well_radar.pdf} & The Well & Biology \ & Surrogate modeling, physics-based prediction & Sequence Prediction/Forecasting & FNO baselines, U-Net baselines \\ \hline
\includegraphics[width=\imageratingswidth]{mmlu_massive_multitask_language_understanding_radar.pdf} & MMLU (Massive Multitask Language Understanding) & Computational Science \ & General reasoning, subject-matter understanding & Generalization & GPT-4o, Gemini 1.5 Pro, o1, DeepSeek-R1 \\ \hline
\includegraphics[width=\imageratingswidth]{satimgnet_radar.pdf} & SatImgNet & Climate \ & Accuracy & Zero-shot land-use classification & 3.83 \\ \hline
\includegraphics[width=\imageratingswidth]{gpqa_diamond_radar.pdf} & GPQA Diamond & Biology \ & Scientific reasoning, deep knowledge & Generalization & o1, DeepSeek-R1 \\ \hline
\includegraphics[width=\imageratingswidth]{prmk_radar.pdf} & PRM800K & Mathematics & Accuracy & Math reasoning and generalization & 3.83 \\ \hline
\includegraphics[width=\imageratingswidth]{feabench_finite_element_analysis_benchmark_evaluating_language_models_on_multiphysics_reasoning_ability_radar.pdf} & FEABench (Finite Element Analysis Benchmark): Evaluating Language Models on Multiphysics Reasoning Ability & Mathematics & Solve time, Error norm & Numerical simulation accuracy and efficiency & 3.83 \\ \hline
\includegraphics[width=\imageratingswidth]{neural_architecture_codesign_for_fast_physics_applications_radar.pdf} & Neural Architecture Codesign for Fast Physics Applications & High Energy Physics & NAC-based BraggNN, NAC-optimized Deep Sets (jet) & Accuracy, Latency, Resource utilization & \cite{weitz2025neuralarchitecturecodesignfast} \\ \hline \\ \hline
\includegraphics[width=\imageratingswidth]{delta_squared-dft_radar.pdf} & Delta Squared-DFT & Chemistry, Materials Science & Delta Squared-ML correction networks, Kernel ridge regression & Mean Absolute Error (eV), Energy ranking accuracy & \cite{khrabrov2024nabla2dftuniversalquantumchemistry} \\ \hline \\ \hline
\includegraphics[width=\imageratingswidth]{hdr_ml_anomaly_challenge_-_sea_level_rise_radar.pdf} & HDR ML Anomaly Challenge - Sea Level Rise & Climate \ & ROC-AUC, Precision/Recall & Detection of environmental anomalies & 3.83 \\ \hline
\includegraphics[width=\imageratingswidth]{vocal_call_locator_vcl_radar.pdf} & Vocal Call Locator (VCL) & Biology \ & Localization error (cm), Recall/Precision & Source localization accuracy in bioacoustic settings & 3.83 \\ \hline
\includegraphics[width=\imageratingswidth]{massspecgym_-_de_novo_molecule_generation_radar.pdf} & MassSpecGym - De novo molecule generation & Chemistry & Graph-based generative models, Retrieval baselines & Structure accuracy, Retrieval precision, Simulation MSE & \cite{neurips2024_c6c31413} \\ \hline \\ \hline
\includegraphics[width=\imageratingswidth]{massspecgym_-_molecule_retrieval_radar.pdf} & MassSpecGym - Molecule Retrieval & Chemistry & Graph-based generative models, Retrieval baselines & Structure accuracy, Retrieval precision, Simulation MSE & \cite{neurips2024_c6c31413} \\ \hline \\ \hline
\includegraphics[width=\imageratingswidth]{massspecgym_-_spectrum_simulation_radar.pdf} & MassSpecGym - Spectrum Simulation & Chemistry & Graph-based generative models, Retrieval baselines & Structure accuracy, Retrieval precision, Simulation MSE & \cite{neurips2024_c6c31413} \\ \hline \\ \hline
\includegraphics[width=\imageratingswidth]{spiqa_scientific_paper_image_question_answering_radar.pdf} & SPIQA (Scientific Paper Image Question Answering) & Computational Science \ & Accuracy, F1 score & Visual-textual reasoning in scientific contexts & 3.67 \\ \hline
\includegraphics[width=\imageratingswidth]{gpqa_a_graduate-level_google-proof_question_and_answer_benchmark_radar.pdf} & GPQA: A Graduate-Level Google-Proof Question and Answer Benchmark & Biology \ & Scientific reasoning, knowledge probing & Generalization & GPT-4 baseline \\ \hline
\includegraphics[width=\imageratingswidth]{medqa_radar.pdf} & MedQA & Biology \ & Medical diagnosis and knowledge retrieval & Generalization & Neural reader, Retrieval-based QA systems \\ \hline
\includegraphics[width=\imageratingswidth]{single_qubit_readout_on_qick_system_radar.pdf} & Single Qubit Readout on QICK System & Computational Science \ & Accuracy, Latency & Single-shot fidelity, inference latency & 3.50 \\ \hline
\includegraphics[width=\imageratingswidth]{cfdbench_fluid_dynamics_radar.pdf} & CFDBench (Fluid Dynamics) & Mathematics & FNO, DeepONet, U-Net & L2 error, MAE & \cite{luo2024cfdbenchlargescalebenchmarkmachine} \\ \hline \\ \hline
\includegraphics[width=\imageratingswidth]{curie_scientific_long-context_understanding_reasoning_and_information_extraction_radar.pdf} & CURIE (Scientific Long-Context Understanding, Reasoning and Information Extraction) & Materials Science, High Energy Physics, Biology \ & Generalization & Reasoning \ & Accuracy \\ \hline
\includegraphics[width=\imageratingswidth]{smart_pixels_for_lhc_radar.pdf} & Smart Pixels for LHC & High Energy Physics & 2-layer pixel NN & Data rejection rate, Power per pixel & \cite{parpillon2024smartpixelsinpixelai} \\ \hline \\ \hline
\includegraphics[width=\imageratingswidth]{lhc_new_physics_dataset_radar.pdf} & LHC New Physics Dataset & High Energy Physics & Autoencoder, Variational autoencoder, Isolation forest & ROC-AUC, Detection efficiency & \cite{https://doi.org/10.5281/zenodo.5046389} \\ \hline \\ \hline
\includegraphics[width=\imageratingswidth]{quantum_computing_benchmarks_qml_radar.pdf} & Quantum Computing Benchmarks (QML) & Computational Science \ & Fidelity, Success probability & Quantum algorithm performance and fidelity & 3.17 \\ \hline
\includegraphics[width=\imageratingswidth]{ultrafast_jet_classification_at_the_hl-lhc_radar.pdf} & Ultrafast jet classification at the HL-LHC & High Energy Physics & MLP, Deep Sets, Interaction Network & Accuracy, Latency, Resource utilization & \cite{odagiu2024ultrafastjetclassificationfpgas} \\ \hline \\ \hline
\includegraphics[width=\imageratingswidth]{hedm_braggnn_radar.pdf} & HEDM (BraggNN) & Materials Science & BraggNN & Localization accuracy, Inference time & \cite{liu2021braggnnfastxraybragg} \\ \hline \\ \hline
\includegraphics[width=\imageratingswidth]{d-stem_radar.pdf} & 4D-STEM & Materials Science & CNN models (prototype) & Classification accuracy, Throughput & \cite{qin2023extremely} \\ \hline \\ \hline
\includegraphics[width=\imageratingswidth]{beam_control_radar.pdf} & Beam Control & High Energy Physics & DDPG, PPO (planned) & Stability, Control loss & \cite{duarte2022fastmlsciencebenchmarksaccelerating3,kafkes2021boostrdatasetacceleratorcontrol} \\ \hline \\ \hline
\includegraphics[width=\imageratingswidth]{intelligent_experiments_through_real-time_ai_radar.pdf} & Intelligent experiments through real-time AI & High Energy Physics & Bipartite Graph Network with Set Transformers (BGN-ST), GarNet (edge-classifier) & Accuracy (charm and beauty detection), Latency (micros), Resource utilization (LUT/FF/BRAM/DSP) & \cite{kvapil2025intelligentexperimentsrealtimeai} \\ \hline \\ \hline
\includegraphics[width=\imageratingswidth]{hdr_ml_anomaly_challenge_-_butterfly_radar.pdf} & HDR ML Anomaly Challenge - Butterfly & Biology \ & Classification accuracy, F1 score & Hybrid detection in biological systems & 3.00 \\ \hline
\includegraphics[width=\imageratingswidth]{dune_radar.pdf} & DUNE & High Energy Physics & CNN, LSTM (planned) & Detection efficiency, Latency & \cite{abud2021deep} \\ \hline \\ \hline
\includegraphics[width=\imageratingswidth]{frontiermath_radar.pdf} & FrontierMath & Mathematics & Accuracy & Symbolic and abstract mathematical reasoning & 2.50 \\ \hline
\includegraphics[width=\imageratingswidth]{aime_american_invitational_mathematics_examination_radar.pdf} & AIME (American Invitational Mathematics Examination) & Mathematics & Accuracy & Mathematical problem-solving and reasoning & 2.33 \\ \hline
\includegraphics[width=\imageratingswidth]{quench_detection_radar.pdf} & Quench detection & High Energy Physics & Autoencoder, RL agents (in development) & ROC-AUC, Detection latency & \cite{quench2024} \\ \hline \\ \hline
\includegraphics[width=\imageratingswidth]{materials_project_radar.pdf} & Materials Project & Materials Science & Automatminer, Crystal Graph Neural Networks & MAE, R{\textasciicircum}2 & \cite{jain2013materials} \\ \hline \\ \hline
\includegraphics[width=\imageratingswidth]{in-situ_high-speed_computer_vision_radar.pdf} & In-Situ High-Speed Computer Vision & High Energy Physics & CNN & Accuracy, FPS & \cite{wei2024lowlatencyopticalbasedmode} \\ \hline \\ \hline
\end{longtable}
}

\end{landscape}

\newlength{\ratingswidth}
\setlength{\ratingswidth}{0.11\textwidth}

\newlength{\imageratingswidth}
\setlength{\imageratingswidth}{0.06\textwidth}

% tighten vertical spacing before/after the longtable
\setlength\LTpre{0pt}
\setlength\LTpost{0pt}
\setlength{\parskip}{0pt}
\setlength{\parindent}{0pt}
\renewcommand{\arraystretch}{0.90}
% tighter horizontal padding and centered m-columns
\setlength{\tabcolsep}{2pt}
\newcolumntype{M}[1]{>{\centering\arraybackslash}m{#1}}

\onecolumn
\begin{landscape}
{\footnotesize
\begin{longtable}{
|M{0.08\textwidth}
|M{0.30\textwidth}
|M{0.20\textwidth}
|M{0.20\textwidth}
|M{0.30\textwidth}
|M{0.11\textwidth}
|}
\caption{Ontology Table for Selected AI Science Benchmarks. \\ 
{\tiny(For detailed view of the Radar Charts, see \cite{www-las-mlcommons-benchmark-collection}.)} } 
\label{tab:ontology} 
\\ \hline
\rowcolor{blue!30} \textbf{Ratings} & \textbf{Name} & \textbf{Domain} & \textbf{Models} & \textbf{Metrics} & \textbf{Citation} \\ \hline
\endfirsthead
\caption{Ontology Table for Selected AI Science Benchmarks (cont.). }\\
\hline
\rowcolor{blue!30} \textbf{Ratings} & \textbf{Name} & \textbf{Domain} & \textbf{Models} & \textbf{Metrics} & \textbf{Citation}  \\ \hline
\endhead
\hline
\multicolumn{6}{r}{Continued on next page} \\ \hline
\endfoot
\hline
\endlastfoot
\includegraphics[width=0.05\textwidth]{climatelearn_-_weather_forcasting_radar.pdf} & ClimateLearn - Weather Forcasting & Climate \& Earth Science & CNN baselines, ResNet variants & RMSE, Anomaly correlation & \cite{nguyen2023climatelearnbenchmarkingmachinelearning} \\ \hline
\includegraphics[width=0.05\textwidth]{climatelearn_-_downscaling_radar.pdf} & ClimateLearn - Downscaling & Climate \& Earth Science & CNN baselines, ResNet variants & RMSE, Anomaly correlation & \cite{nguyen2023climatelearnbenchmarkingmachinelearning} \\ \hline
\includegraphics[width=0.05\textwidth]{climatelearn_-_climate_projection_radar.pdf} & ClimateLearn - Climate Projection & Climate \& Earth Science & CNN baselines, ResNet variants & RMSE, Anomaly correlation & \cite{nguyen2023climatelearnbenchmarkingmachinelearning} \\ \hline
\includegraphics[width=0.05\textwidth]{mlcommons_science_-_cloudmask_radar.pdf} & MLCommons Science - CloudMask & Climate \& Earth Science & CNN, GNN, Transformer & MAE, Accuracy, Speedup vs simulation & \cite{10.1007/978-3-031-23220-6_4} \\ \hline
\includegraphics[width=0.05\textwidth]{mlcommons_science_-_earthquake_radar.pdf} & MLCommons Science - Earthquake & Climate \& Earth Science & CNN, GNN, Transformer & MAE, Accuracy, Speedup vs simulation & \cite{10.1007/978-3-031-23220-6_4} \\ \hline
\includegraphics[width=0.05\textwidth]{mlcommons_science_-_candle_uno_radar.pdf} & MLCommons Science - Candle UNO & Biology \& Medicine & CNN, GNN, Transformer & MAE, Accuracy, Speedup vs simulation & \cite{10.1007/978-3-031-23220-6_4} \\ \hline
\includegraphics[width=0.05\textwidth]{mlcommons_science_-_stemdl_radar.pdf} & MLCommons Science - STEMDL & Materials Science & CNN, GNN, Transformer & MAE, Accuracy, Speedup vs simulation & \cite{10.1007/978-3-031-23220-6_4} \\ \hline
\includegraphics[width=0.05\textwidth]{arc-challenge_advanced_reasoning_challenge_radar.pdf} & ARC-Challenge (Advanced Reasoning Challenge) & Computational Science \& AI & GPT-4, Claude & Accuracy & \cite{allenai:arc} \\ \hline
\includegraphics[width=0.05\textwidth]{molgen_radar.pdf} & MOLGEN & Chemistry & MolGen & Validity\%, Novelty\%, QED, Docking score, penalized logP & \cite{fang2024domainagnosticmoleculargenerationchemical} \\ \hline
\includegraphics[width=0.05\textwidth]{open_graph_benchmark_ogb_-_biology_radar.pdf} & Open Graph Benchmark (OGB) - Biology & Biology \& Medicine & GCN, GraphSAGE, GAT & Accuracy, ROC-AUC & \cite{hu2021opengraphbenchmarkdatasets} \\ \hline
\includegraphics[width=0.05\textwidth]{llms_for_crop_science_radar.pdf} & LLMs for Crop Science & Climate \& Earth Science & GPT-3.5, GPT-4, Claude-3-opus, Qwen-max, LLama3-8B, InternLM2-7B, Qwen1.5-7B & Accuracy, F1 score & \cite{zhang2024empowering} \\ \hline
\includegraphics[width=0.05\textwidth]{scicode_radar.pdf} & SciCode & Computational Science \& AI & Claude3.5-Sonnet & Solve rate (\%) & \cite{tian2024scicoderesearchcodingbenchmark} \\ \hline
\includegraphics[width=0.05\textwidth]{calochallenge__radar.pdf} & CaloChallenge 2022 & High Energy Physics & VAE variants, GAN variants, Normalizing flows, Diffusion models & Histogram similarity, Classifier AUC, Generation latency & \cite{krause2024calochallenge2022communitychallenge} \\ \hline
\includegraphics[width=0.05\textwidth]{pdebench_radar.pdf} & PDEBench & Computational Science \& AI, Climate \& Earth Science, Mathematics & FNO, U-Net, PINN, Gradient-Based inverse methods & RMSE, boundary RMSE, Fourier RMSE & \cite{takamoto2024pdebenchextensivebenchmarkscientific} \\ \hline
\includegraphics[width=0.05\textwidth]{urban_data_layer_udl_-_pm_concentration_prediction_radar.pdf} & Urban Data Layer (UDL) - PM2.5 Concentration Prediction & Climate \& Earth Science & Baseline regression/classification pipelines & Task-specific accuracy or RMSE & \cite{neurips2024_0db7f135} \\ \hline
\includegraphics[width=0.05\textwidth]{urban_data_layer_udl_-_built-up_area_classification_radar.pdf} & Urban Data Layer (UDL) - Built-up Area Classification & Climate \& Earth Science & Baseline regression/classification pipelines & Task-specific accuracy or RMSE & \cite{neurips2024_0db7f135} \\ \hline
\includegraphics[width=0.05\textwidth]{urban_data_layer_udl_-_administrative_boundaries_identification_radar.pdf} & Urban Data Layer (UDL) - Administrative Boundaries Identification & Climate \& Earth Science & Baseline regression/classification pipelines & Task-specific accuracy or RMSE & \cite{neurips2024_0db7f135} \\ \hline
\includegraphics[width=0.05\textwidth]{urban_data_layer_udl_-_el_nino_anomaly_detection_radar.pdf} & Urban Data Layer (UDL) - El Nino Anomaly Detection & Climate \& Earth Science & Baseline regression/classification pipelines & Task-specific accuracy or RMSE & \cite{neurips2024_0db7f135} \\ \hline
\includegraphics[width=0.05\textwidth]{spiqa_llm_radar.pdf} & SPIQA (LLM) & Computational Science \& AI & LLaVA, MiniGPT-4, Owl-LLM adapter variants & Accuracy, F1 score & \cite{pramanick2025spiqadatasetmultimodalquestion} \\ \hline
\includegraphics[width=0.05\textwidth]{mlcommons_medical_ai_-_pancreas_segmentation_dfci_radar.pdf} & MLCommons Medical AI - Pancreas Segmentation (DFCI) & Biology \& Medicine & MedPerf-validated CNNs, GaNDLF workflows & ROC AUC, Accuracy, Fairness metrics & \cite{karargyris2023federated} \\ \hline
\includegraphics[width=0.05\textwidth]{mlcommons_medical_ai_-_brain_tumor_segmentation_brats_radar.pdf} & MLCommons Medical AI - Brain Tumor Segmentation (BraTS) & Biology \& Medicine & MedPerf-validated CNNs, GaNDLF workflows & ROC AUC, Accuracy, Fairness metrics & \cite{karargyris2023federated} \\ \hline
\includegraphics[width=0.05\textwidth]{mlcommons_medical_ai_-__surgical_workflow_phase_recognition_surgmlcube_radar.pdf} & MLCommons Medical AI -  Surgical Workflow Phase Recognition (SurgMLCube) & Biology \& Medicine & MedPerf-validated CNNs, GaNDLF workflows & ROC AUC, Accuracy, Fairness metrics & \cite{karargyris2023federated} \\ \hline
\includegraphics[width=0.05\textwidth]{seafloorai_radar.pdf} & SeafloorAI & Climate \& Earth Science & SegFormer, ViLT-style multimodal models & Segmentation pixel accuracy, QA accuracy & \cite{nguyen2024seafloor} \\ \hline
\includegraphics[width=0.05\textwidth]{seafloorgenai_radar.pdf} & SeafloorGenAI & Climate \& Earth Science & SegFormer, ViLT-style multimodal models & Segmentation pixel accuracy, QA accuracy & \cite{nguyen2024seafloor} \\ \hline
\includegraphics[width=0.05\textwidth]{gess_-_track_pileup_radar.pdf} & GeSS - Track Pileup & High Energy Physics & GCN, EGNN, DimeNet++ & Accuracy, RMSE, OOD robustness delta & \cite{neurips2024_a8063075} \\ \hline
\includegraphics[width=0.05\textwidth]{gess_-_track_signal_radar.pdf} & GeSS - Track Signal & High Energy Physics & GCN, EGNN, DimeNet++ & Accuracy, RMSE, OOD robustness delta & \cite{neurips2024_a8063075} \\ \hline
\includegraphics[width=0.05\textwidth]{gess_-_drugood_radar.pdf} & GeSS - DrugOOD & Biology \& Medicine & GCN, EGNN, DimeNet++ & Accuracy, RMSE, OOD robustness delta & \cite{neurips2024_a8063075} \\ \hline
\includegraphics[width=0.05\textwidth]{gess_-_qmof_radar.pdf} & GeSS - QMOF & Materials Science & GCN, EGNN, DimeNet++ & Accuracy, RMSE, OOD robustness delta & \cite{neurips2024_a8063075} \\ \hline
\includegraphics[width=0.05\textwidth]{ocp_open_catalyst_project_radar.pdf} & OCP (Open Catalyst Project) & Chemistry, Materials Science & CGCNN, SchNet, DimeNet++, GemNet-OC & MAE (energy), MAE (force) & \cite{chanussot2021oc20,tran2023oc22} \\ \hline
\includegraphics[width=0.05\textwidth]{jet_classification_radar.pdf} & Jet Classification & High Energy Physics & Keras DNN, QKeras quantized DNN & Accuracy, AUC & \cite{duarte2022fastml} \\ \hline
\includegraphics[width=0.05\textwidth]{irregular_sensor_data_compression_radar.pdf} & Irregular Sensor Data Compression & High Energy Physics & Autoencoder, Quantized autoencoder & MSE, Compression ratio & \cite{duarte2022fastml} \\ \hline
\includegraphics[width=0.05\textwidth]{mlperf_hpc_-_cosmoflow_radar.pdf} & MLPerf HPC - Cosmoflow & High Energy Physics & CosmoFlow, DeepCAM, OpenCatalyst & Training time, Accuracy, GPU utilization & \cite{farrell2021mlperfhpcholisticbenchmark} \\ \hline
\includegraphics[width=0.05\textwidth]{mlperf_hpc_-_deepcam_radar.pdf} & MLPerf HPC - DeepCAM & Climate \& Earth Science & DeepCAM & Training time, Accuracy, GPU utilization & \cite{farrell2021mlperfhpcholisticbenchmark} \\ \hline
\includegraphics[width=0.05\textwidth]{mlperf_hpc_-_open_catalyst_project_dimenet__radar.pdf} & MLPerf HPC - Open Catalyst Project DimeNet++  & Chemistry & DeepCAM & Training time, Accuracy, GPU utilization & \cite{farrell2021mlperfhpcholisticbenchmark} \\ \hline
\includegraphics[width=0.05\textwidth]{mlperf_hpc_-_openfold_radar.pdf} & MLPerf HPC - OpenFold & Biology \& Medicine & DeepCAM & Training time, Accuracy, GPU utilization & \cite{farrell2021mlperfhpcholisticbenchmark} \\ \hline
\includegraphics[width=0.05\textwidth]{hdr_ml_anomaly_challenge_-_gravitational_waves_radar.pdf} & HDR ML Anomaly Challenge - Gravitational Waves & High Energy Physics & Deep latent CNNs, Autoencoders & ROC-AUC, Precision/Recall & \cite{campolongo2025buildingmachinelearningchallenges} \\ \hline
\includegraphics[width=0.05\textwidth]{supercond_-_property_prediction_radar.pdf} & SuperCon3D - Property Prediction & Materials Science & SODNet, DiffCSP-SC & MAE (Tc), Validity of generated structures & \cite{neurips2024_c4e3b55e} \\ \hline
\includegraphics[width=0.05\textwidth]{supercond_-__inverse_crystal_structure_generation_radar.pdf} & SuperCon3D -  Inverse Crystal Structure Generation & Materials Science & SODNet, DiffCSP-SC & MAE (Tc), Validity of generated structures & \cite{neurips2024_c4e3b55e} \\ \hline
\includegraphics[width=0.05\textwidth]{baisbench_biological_ai_scientist_benchmark_-_question_answering_radar.pdf} & BaisBench (Biological AI Scientist Benchmark) - Question Answering & Biology \& Medicine & LLM-based AI scientist agents & Annotation accuracy, QA accuracy & \cite{luo2025benchmarkingaiscientistsomics} \\ \hline
\includegraphics[width=0.05\textwidth]{baisbench_biological_ai_scientist_benchmark_-_cell_type_annotation_radar.pdf} & BaisBench (Biological AI Scientist Benchmark) - Cell Type Annotation & Biology \& Medicine & LLM-based AI scientist agents & Annotation accuracy, QA accuracy & \cite{luo2025benchmarkingaiscientistsomics} \\ \hline
\includegraphics[width=0.05\textwidth]{the_well_radar.pdf} & The Well & Biology \& Medicine, Computational Science \& AI, High Energy Physics & FNO baselines, U-Net baselines & Dataset size, Domain breadth & \cite{neurips2024_4f9a5acd} \\ \hline
\includegraphics[width=0.05\textwidth]{mmlu_massive_multitask_language_understanding_radar.pdf} & MMLU (Massive Multitask Language Understanding) & Computational Science \& AI & GPT-4o, Gemini 1.5 Pro, o1, DeepSeek-R1 & Accuracy & \cite{hendrycks2021measuring} \\ \hline
\includegraphics[width=0.05\textwidth]{satimgnet_radar.pdf} & SatImgNet & Climate \& Earth Science & CLIP, BLIP, ALBEF & Accuracy & \cite{roberts2023satin} \\ \hline
\includegraphics[width=0.05\textwidth]{gpqa_diamond_radar.pdf} & GPQA Diamond & Biology \& Medicine, Chemistry, High Energy Physics & o1, DeepSeek-R1 & Accuracy & \cite{rein2023gpqagraduatelevelgoogleproofqa} \\ \hline
\includegraphics[width=0.05\textwidth]{prmk_radar.pdf} & PRM800K & Mathematics & GPT-4 & Accuracy & \cite{lightman2023lets} \\ \hline
\includegraphics[width=0.05\textwidth]{feabench_finite_element_analysis_benchmark_evaluating_language_models_on_multiphysics_reasoning_ability_radar.pdf} & FEABench (Finite Element Analysis Benchmark): Evaluating Language Models on Multiphysics Reasoning Ability & Mathematics & FEniCS, deal.II & Solve time, Error norm & \cite{mudur2025feabenchevaluatinglanguagemodels} \\ \hline
\includegraphics[width=0.05\textwidth]{neural_architecture_codesign_for_fast_physics_applications_radar.pdf} & Neural Architecture Codesign for Fast Physics Applications & High Energy Physics & NAC-based BraggNN, NAC-optimized Deep Sets (jet) & Accuracy, Latency, Resource utilization & \cite{weitz2025neuralarchitecturecodesignfast} \\ \hline
\includegraphics[width=0.05\textwidth]{delta_squared-dft_radar.pdf} & Delta Squared-DFT & Chemistry, Materials Science & Delta Squared-ML correction networks, Kernel ridge regression & Mean Absolute Error (eV), Energy ranking accuracy & \cite{khrabrov2024nabla2dftuniversalquantumchemistry} \\ \hline
\includegraphics[width=0.05\textwidth]{hdr_ml_anomaly_challenge_-_sea_level_rise_radar.pdf} & HDR ML Anomaly Challenge - Sea Level Rise & Climate \& Earth Science & CNNs, RNNs, Transformers & ROC-AUC, Precision/Recall & \cite{campolongo2025buildingmachinelearningchallenges} \\ \hline
\includegraphics[width=0.05\textwidth]{vocal_call_locator_vcl_radar.pdf} & Vocal Call Locator (VCL) & Biology \& Medicine & CNN-based SSL models & Localization error (cm), Recall/Precision & \cite{neurips2024_c00d37d6} \\ \hline
\includegraphics[width=0.05\textwidth]{massspecgym_-_de_novo_molecule_generation_radar.pdf} & MassSpecGym - De novo molecule generation & Chemistry & Graph-based generative models, Retrieval baselines & Structure accuracy, Retrieval precision, Simulation MSE & \cite{neurips2024_c6c31413} \\ \hline
\includegraphics[width=0.05\textwidth]{massspecgym_-_molecule_retrieval_radar.pdf} & MassSpecGym - Molecule Retrieval & Chemistry & Graph-based generative models, Retrieval baselines & Structure accuracy, Retrieval precision, Simulation MSE & \cite{neurips2024_c6c31413} \\ \hline
\includegraphics[width=0.05\textwidth]{massspecgym_-_spectrum_simulation_radar.pdf} & MassSpecGym - Spectrum Simulation & Chemistry & Graph-based generative models, Retrieval baselines & Structure accuracy, Retrieval precision, Simulation MSE & \cite{neurips2024_c6c31413} \\ \hline
\includegraphics[width=0.05\textwidth]{spiqa_scientific_paper_image_question_answering_radar.pdf} & SPIQA (Scientific Paper Image Question Answering) & Computational Science \& AI & Chain-of-Thought models, Multimodal QA systems & Accuracy, F1 score & \cite{zhong2024spiqa} \\ \hline
\includegraphics[width=0.05\textwidth]{gpqa_a_graduate-level_google-proof_question_and_answer_benchmark_radar.pdf} & GPQA: A Graduate-Level Google-Proof Question and Answer Benchmark & Biology \& Medicine, High Energy Physics, Chemistry & GPT-4 baseline & Accuracy & \cite{rein2023gpqagraduatelevelgoogleproofqa2} \\ \hline
\includegraphics[width=0.05\textwidth]{medqa_radar.pdf} & MedQA & Biology \& Medicine & Neural reader, Retrieval-based QA systems & Accuracy & \cite{jin2020diseasedoespatienthave} \\ \hline
\includegraphics[width=0.05\textwidth]{single_qubit_readout_on_qick_system_radar.pdf} & Single Qubit Readout on QICK System & Computational Science \& AI & hls4ml quantized NN & Accuracy, Latency & \cite{diguglielmo2025endtoendworkflowmachinelearningbased} \\ \hline
\includegraphics[width=0.05\textwidth]{cfdbench_fluid_dynamics_radar.pdf} & CFDBench (Fluid Dynamics) & Mathematics & FNO, DeepONet, U-Net & L2 error, MAE & \cite{luo2024cfdbenchlargescalebenchmarkmachine} \\ \hline
\includegraphics[width=0.05\textwidth]{curie_scientific_long-context_understanding_reasoning_and_information_extraction_radar.pdf} & CURIE (Scientific Long-Context Understanding, Reasoning and Information Extraction) & Materials Science, High Energy Physics, Biology \& Medicine, Chemistry, Climate \& Earth Science & unkown & Accuracy & \cite{cui2025curieevaluatingllmsmultitask} \\ \hline
\includegraphics[width=0.05\textwidth]{smart_pixels_for_lhc_radar.pdf} & Smart Pixels for LHC & High Energy Physics & 2-layer pixel NN & Data rejection rate, Power per pixel & \cite{parpillon2024smartpixelsinpixelai} \\ \hline
\includegraphics[width=0.05\textwidth]{lhc_new_physics_dataset_radar.pdf} & LHC New Physics Dataset & High Energy Physics & Autoencoder, Variational autoencoder, Isolation forest & ROC-AUC, Detection efficiency & \cite{https://doi.org/10.5281/zenodo.5046389} \\ \hline
\includegraphics[width=0.05\textwidth]{quantum_computing_benchmarks_qml_radar.pdf} & Quantum Computing Benchmarks (QML) & Computational Science \& AI & IBM Q, IonQ, AQT@LBNL & Fidelity, Success probability & \cite{bowles2024betterclassicalsubtleart} \\ \hline
\includegraphics[width=0.05\textwidth]{ultrafast_jet_classification_at_the_hl-lhc_radar.pdf} & Ultrafast jet classification at the HL-LHC & High Energy Physics & MLP, Deep Sets, Interaction Network & Accuracy, Latency, Resource utilization & \cite{odagiu2024ultrafastjetclassificationfpgas} \\ \hline
\includegraphics[width=0.05\textwidth]{hedm_braggnn_radar.pdf} & HEDM (BraggNN) & Materials Science & BraggNN & Localization accuracy, Inference time & \cite{liu2021braggnnfastxraybragg} \\ \hline
\includegraphics[width=0.05\textwidth]{d-stem_radar.pdf} & 4D-STEM & Materials Science & CNN models (prototype) & Classification accuracy, Throughput & \cite{qin2023extremely} \\ \hline
\includegraphics[width=0.05\textwidth]{beam_control_radar.pdf} & Beam Control & High Energy Physics & DDPG, PPO (planned) & Stability, Control loss & \cite{duarte2022fastml,kafkes2021boostrdatasetacceleratorcontrol} \\ \hline
\includegraphics[width=0.05\textwidth]{intelligent_experiments_through_real-time_ai_radar.pdf} & Intelligent experiments through real-time AI & High Energy Physics & Bipartite Graph Network with Set Transformers (BGN-ST), GarNet (edge-classifier) & Accuracy (charm and beauty detection), Latency (micros), Resource utilization (LUT/FF/BRAM/DSP) & \cite{kvapil2025intelligentexperimentsrealtimeai} \\ \hline
\includegraphics[width=0.05\textwidth]{hdr_ml_anomaly_challenge_-_butterfly_radar.pdf} & HDR ML Anomaly Challenge - Butterfly & Biology \& Medicine & CNN-based detectors & Classification accuracy, F1 score & \cite{campolongo2025buildingmachinelearningchallenges} \\ \hline
\includegraphics[width=0.05\textwidth]{dune_radar.pdf} & DUNE & High Energy Physics & CNN, LSTM (planned) & Detection efficiency, Latency & \cite{abud2021deep} \\ \hline
\includegraphics[width=0.05\textwidth]{frontiermath_radar.pdf} & FrontierMath & Mathematics & unknown & Accuracy & \cite{glazer2024frontiermathbenchmarkevaluatingadvanced} \\ \hline
\includegraphics[width=0.05\textwidth]{aime_american_invitational_mathematics_examination_radar.pdf} & AIME (American Invitational Mathematics Examination) & Mathematics & unknown & Accuracy & \cite{www-aime} \\ \hline
\includegraphics[width=0.05\textwidth]{quench_detection_radar.pdf} & Quench detection & High Energy Physics & Autoencoder, RL agents (in development) & ROC-AUC, Detection latency & \cite{quench2024} \\ \hline
\includegraphics[width=0.05\textwidth]{materials_project_radar.pdf} & Materials Project & Materials Science & Automatminer, Crystal Graph Neural Networks & MAE, R{\textasciicircum}2 & \cite{jain2013materials} \\ \hline
\includegraphics[width=0.05\textwidth]{in-situ_high-speed_computer_vision_radar.pdf} & In-Situ High-Speed Computer Vision & High Energy Physics & CNN & Accuracy, FPS & \cite{wei2024lowlatencyopticalbasedmode} \\ \hline
\end{longtable}
}

\end{landscape}
\twocolumn



% \providecommand{\normal}[1]{#1}

\begin{table*}[htbp]
\centering
\caption{MLCommons Benchmark Suites and Included Workloads (v5.x, 2025)}
\begin{tabular}{|p{0.2\textwidth}|p{0.2\textwidth}|p{0.5\textwidth}|}
\hline
\rowcolor{blue!30} \textbf{MLPerf Suite} & \textbf{Category} & \textbf{Workloads / Models Included} \\ 
\hline
\hline
\normal{Inference: Datacenter} & Vision, Language, Recommendation, GNN, Audio & 
ResNet50-v1.5 (Image Classification); RetinaNet (Object Detection); 3D-UNet (Medical Segmentation); Stable Diffusion / SDXL (Text-to-Image); DLRM-DCNv2 (Recommendation); BERT-Large (QA); LLaMA 2 70B / LLaMA 3.1 405B / Mixtral 8×7B / GPT-J (LLMs, Summarization, QA); Whisper / RNN-T (Speech); R-GAT (GNN); PointPainting (3D Object Detection) \\ 
\hline
\normal{Training} & Vision, Language, Recommendation, Diffusion, Graph, RL & 
ResNet-50 (Classification); Mask R-CNN (Detection); 3D U-Net (Segmentation); DLRM/DCNv2 (Recommendation); LLaMA 2 70B, LLaMA 3.1 8B/405B (LLMs); Stable Diffusion v2 / FLUX.1 (Image Generation); R-GAT (Graph NN); RNN-T (Speech); Mini-Go (Reinforcement Learning) \\ 
\hline
\normal{Inference: Edge} & Edge AI / On-device Vision and NLP & 
MobileNet V1–V4 (Classification); SSD-MobileNet (Detection); On-device NLP and Vision tasks for latency/power benchmarking \\ 
\hline
\normal{Inference: Mobile} & Mobile AI & 
MobileNetV4-Conv-L (Classification); Mobile SSD variants (Detection); Small-scale on-device inference benchmarks for smartphones/tablets \\ 
\hline
\normal{Inference: Tiny} & TinyML / MCU workloads & 
Image Classification (TinyML models); Audio Keyword Spotting; Visual Wake Words; Motion Classification (MCU-friendly models) \\ 
\hline
\normal{Client} & Client-side LLMs & 
LLaMA 2 7B Chat; LLaMA 3.1 8B Instruct; Phi 3.5 Mini Instruct; Phi 4 Reasoning 14B (Code, Reasoning, Summarization tasks) \\ 
\hline
\normal{Storage} & Storage and Data Pipeline & 
Workloads simulate data ingestion, preprocessing, and I/O for Vision (ResNet), NLP (BERT), and Recommendation (DLRM) pipelines \\ 
\hline
\normal{AlgoPerf} & Algorithm Efficiency & 
ResNet-50 (ImageNet); ViT (Vision Transformer); U-Net (MRI reconstruction); DLRMsmall (Criteo CTR); DeepSpeech / Conformer (Speech); OGBG (Graph); plus other algorithmic testbeds (8 total workloads) \\ 
\hline
\normal{Automotive} & Autonomous Driving & 
PointPainting (3D Object Detection, Waymo Dataset) \\ 
\hline
\normal{AILuminate} & AI Safety / Robustness & 
Safety, Jailbreak, and Risk Evaluation workloads across LLMs (AILuminate v0.5 benchmark) \\ 
\hline
\normal{HPC} & & \\ 
\hline
\normal{Science} & & \\
\hline
\end{tabular}
\end{table*}


\subsection{Technical aspects of AI Benchmarks}

In addition to discoverability challenges, there are also technical issues that need to be addressed in dealing with democratization and AI benchmark carpentry.


\subsubsection{Workflows}
\label{sec:benchmarks-mlcommons-desktop}

There are many workflow frameworks that can support the AI Benchmark Workflow. Two of them 
are the Compute Coordinator and the Experiment Executer; they can be used in conjunction or separately \cite{las-2022-templated}. The Compute Coordinator allows hybrid infrastructure access from the benchmark application, while the Experiment Executor allows the repeated execution of templated benchmarks. Both produce results in a structured fashion so they can be combined from multiple experiments and multiple infrastructures in order to support the FAIR principles.

\subsubsection{Containerization}
\label{sec:benchmarks-mlcommons-hpc}

Benchmarking on HPC and even smaller machines can be simplified by providing containerized environments which not only enable easy deployment, but also can harmonize execution by providing stable operating system and software environments. In addition to portable makefiles, the uniform generation of containers can be leveraged between applications.
Although docker is today widely used to containerize applications, on HPC systems we find that limited root access on many HPC systems led to the development of apptainers.
Hence, AI benchmarking carpentry should include the development of software in apptainers directly or converting Docker containers to apptainers.

\subsubsection{System-Dependent Software and Deployment Variability}

Benchmarking can be complex if the software, libraries and infrastructure differ across systems.
To support coordinated  benchmarking across different machines, we have introduced a 
templates hybrid reusable computational analytics workflow management framework with cloudmesh. This framework has been be applied to multiple Deep Learning MLCommons Applications. The details are explained in  \cite{las-2022-templated}. Utilizing such workflow systems promotes adaptation as deployment and execution is typically included in the workflow specifications. However, it can also address adaptation and modifications to future improvements and porting to different hardware as a working template is already provided..
    
\subsubsection{Logging and Monitoring}
\label{sec:benchmarks-mlcommons-logging}

A variety of logging frameworks exist for AI Benchmark logging. This includes logging tools such as MLPerf logging. While such tools provide elementary logging features, their outputs are not human readable and require post processing. This is also an issue when running applications in interactive mode during debugging phases. For this reason, we have provided Cloudmesh-stopwatch that not only allows human readable format, but also allows automatic MLPerf logging (if desired) with a single line change in the code. Cloudmesh stopwatch supports Python, shell, and batch script execution, and employs a consistent log format across all three. 

In general, we distinguish between four types of monitoring: (a) Infrastructure Monitoring, (b) Application  Monitoring, (c) Training Monitoring, and (d) Model-Level Monitoring. A wide range of tools exists for each type, making it essential to identify those that provide effective functionality while remaining easy to use. TensorBoard is one example.
